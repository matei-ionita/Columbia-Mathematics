\documentclass[12 pt]{article}
\usepackage{amsmath,amssymb,amsthm,fullpage,amsfonts,enumerate,textcomp, eurosym, tikz-cd, fullpage}
\title{Ruling of lines on a quadric surface}
\author{Matei Ionita}

\newcommand{\R}{\mathbb{R}}
\newcommand{\Q}{\mathbb{Q}}
\newcommand{\Z}{\mathbb{Z}}
\newcommand{\F}{\mathbb{F}}
\newcommand{\C}{\mathbb{C}}
\newcommand{\CP}{\mathbb{C}\mathbb{P}}
\newcommand{\RP}{\mathbb{R}\mathbb{P}}
\newcommand{\Proj}{\mathbb{P}}
\newcommand{\N}{\mathbb{N}}
\newcommand{\p}{\partial}
\newcommand{\fr}{\mathfrak}

\DeclareMathOperator{\Mor}{Mor}
\DeclareMathOperator{\Hom}{Hom}
\DeclareMathOperator{\length}{length}
\DeclareMathOperator{\res}{Res}
\DeclareMathOperator{\Int}{Int}
\DeclareMathOperator{\Ext}{Ext}
\DeclareMathOperator{\Aut}{Aut}
\DeclareMathOperator{\Gal}{Gal}
\DeclareMathOperator{\Sym}{Sym}
\DeclareMathOperator{\Lie}{Lie}
\DeclareMathOperator{\id}{Id}
\DeclareMathOperator{\tr}{tr}
\DeclareMathOperator{\irr}{irr}
\DeclareMathOperator{\supp}{supp}
\DeclareMathOperator{\trdeg}{trdeg}
\DeclareMathOperator{\Spec}{Spec}
\DeclareMathOperator{\Nm}{Nm}
\DeclareMathOperator{\ord}{ord}
\DeclareMathOperator{\imag}{Im}

\begin{document}
  \maketitle

\section{Roadmap}
Today I will talk about quadric surfaces in $\Proj^3$. As the name suggests, a quadric surface is the zero locus of a quadratic homogenous
polynomial. As usual, it's more convenient to work over the field of complex numbers, as this will allow us to easily classify all quadric surfaces.
In particular, we will see that all nonsingular quadrics are isomorphic. Using this result, it will be easy to prove the main theorem of the day. It
says that, on each nonsingular quadric, there exist two infinite families of lines, with certain intersection properties. After this, I will briefly talk
about another way in which we can count the lines, which is faster, but doesn't allow us to explicitly find equations for the embedded lines.

\section{Classification of quadrics}
In $\Proj^3$, a quadric is given by a vanishing of a polynomial which, in general, is pretty disgusting:
\[	a X_0^2 + bX_1^2 + cX_2^2 + dX_3^2 + e X_0 X_1 + fX_0 X_2 + g X_0 X_3 + hX_1X_2 + iX_1X_3 + jX_2X_3 = 0	\]
The natural thing to try is to perform a linear change of variables that brings this to a much simpler form. We have already seen this procedure
before, in Robert's talk, when we showed that all conics in $\Proj^2$ are isomorphic, and this proof uses the same argument. We begin by
writing the above equation in matrix form:
\[	\left( \begin{array} {cccc}  X_0 & X_1 & X_2 & X_3  \end{array} \right)
	\left( \begin{array} {cccc}  a & e/2 & f/2 & g/2 \\ e/2 & b& h/2 & i/2 \\ f/2 & h/2 & c & j/2 \\ g/2 & i/2 & j/2 & d \end{array} \right)
	\left( \begin{array} {c}  X_0 \\ X_1 \\ X_2 \\ X_3 \end{array} \right)  = 0	\]
The matrix that we obtained is symmetric, therefore it can be written in the form $O^T D O$, where $O$ is orthogonal and $D$ is diagonal.
Thus, acting by $O$ on the vector of old variables gives us the new variables, which by abuse of language we also call $X_0, \dots, X_3$. We
we have reduced our quadratic polynomial to:
\[	f(X_0, X_1, X_2, X_3) = a X_0^2 + bX_1^2 + cX_2^2 + dX_3^2 = 0	\]
Provided that $a \neq 0$, we can let $\sqrt{a} X_0 \to X_0$ in order to incorporate the constant into the variable, and similarly for the other
constants. Therefore we're only interested in whether the constants are zero or not. From this, we conclude that there are four isomorphism classes of
quadrics, with one, two, three or four nonzero constants in the polynomial $f$. The latter is the class of nonsingular quadrics: it's easy to see
that the $1\times 4$ matrix of partial derivatives of $f$ has full rank at all points of the zero locus iff all the constants are nonzero.
\[             Df = \left( \begin{array} {cccc} 2aX_0 & 2bX_1 & 2cX_2 & 2dX_3 \end{array} \right)            \]
Indeed, if $a,b,c,d \neq 0$, then the only way for $Df$ to have rank 0 would be $X_0 = X_1 = X_2 = X_3 = 0$, which does not belong to $\Proj^3$.
However, if, say, $a = 0$, then $[1:0:0:0]$ is a singular point.

Let's take each of these four classes and discuss it briefly. In the figure below you can see an example of a nonsingular quadric, which we will use 
in our following discussion of embedded lines. Whatever statement we prove about lines on this surface, it will of course be true for all 
nonsingular quadrics.
\begin{figure}[h!]
  \caption{Nonsingular quadric, given by equation $X_0X_1 - X_2X_3 =0$.}
  \centering
    \includegraphics[width=0.5\textwidth]{Hyperbolic-paraboloid}
\end{figure}

If $a=0$, but $b,c,d \neq 0$, we are left with an equation of the type $X_1^2+ X_2^2 - X_3^2 = 0$, whose real points form a cone. The singular point
$[1:0:0:0]$ is the vertex of the cone. If $a, b =0$ and $c,d \neq 0$, the quadratic is reducible:
\[ X_2^2 - X_3^2 = (X_2 - X_3)(X_2 + X_3) = 0 \]
Therefore the surface is the union of the two hyperplanes $X_2 - X_3 = 0, X_2 + X_3 = 0$. Finally, the fourth class is a bit silly. If only $d\neq 0$, we
have $X_3^2 = 0$, which simply gives the hyperplane $X_3 = 0$ with multiplicity 2.


\section{The Segre embedding}
Before counting lines on quadrics, it's useful to introduce a preliminary construction called Segre embedding. It is a map $\psi : \Proj^m \times
\Proj^n \to \Proj^{(n+1)(m+1) - 1}$ given by:
\[      \big( ( X_0, \dots, X_n) , (Y_0, \dots, Y_m  )\big)  \mapsto (X_i Y_j)    \]
In order to visualize this, we can arrange the homogenous coordinates of $\Proj^{(n+1)(m+1) - 1}$ into an array as follows:
\[  \left(  \begin{array} {ccc} Z_{00} &\cdots & Z_{0m} \\ \vdots & \ddots & \vdots \\ Z_{n0} &\cdots & Z_{nm} \end{array} \right) \]
Then the image of the Segre embedding is:
\[  \left(  \begin{array} {ccc} X_0 Y_0 &\cdots & X_0 Y_m \\ \vdots & \ddots & \vdots \\ X_n Y_0 &\cdots & X_nY_m \end{array} \right) \]
The dimension of $\psi(\Proj^m \times \Proj^n)$ is $m+n$, which is generally much smaller than that of $\Proj^{(n+1)(m+1) - 1}$. More
concretely, we can characterize the image of $\psi$ as the subspace of $\Proj^{(n+1)(m+1) - 1}$ where
\[  \text{rank} \left(  \begin{array} {ccc} Z_{00} &\cdots & Z_{0m} \\ \vdots & \ddots & \vdots \\ Z_{n0} &\cdots & Z_{nm} \end{array} \right)
\leq 1 \]
In other words, $\left|  \begin{array} {cc} Z_{ik} & Z_{il} \\ Z_{jk} & Z_{jl} \end{array} \right| = 0$, for all $0 \leq i,j \leq n$ and all $0\leq k,l\leq m$.


\section{Line counting}
The concept of a line seems self-explanatory, but it may be worth it to define a line in projective space. We take a line in $\Proj^3$ to be the image
of a nonzero linear map $\phi: \Proj^1 \to \Proj^3$. The most general line therefore looks like:
\[      [t:s] \mapsto [at + bs : ct+ds : et+fs : gt+hs]    \]
(There exists another way to define lines, which is more easily generalizable to linar variaties of higher dimensions. A line in $\Proj^n$ can 
be taken to be the locus of simultaneous vanishing of $n-1$ linearly independent homogenous, linear polynomials. It's not hard to see that 
these definitions are equivalent, so in this talk I'll just use the first one, which is easier to operate with.) The space of lines in $\Proj^3$ 
is pretty large, 7-dimensional in fact. We are interested in describing those lines that are contained in a given nonsingular quadric, and 
since all such quadrics are isomorphic, it suffices to analyze the one given by $XY - ZW = 0$, which was pictured in Figure 1.

Describing lines contained in a surface has an important practical application. Quadrics are used in architecture, for constructions such as 
saddle roofs and cooling towers. Generally, curved surfaces are more difficult to construct than flat ones, and they also need more complicated 
reinforcement to keep them from collapsing. However, if enough straight lines are contained in the curved surface, engineers can first
construct a skeleton out of straight steel rods, and then add the rest of the material around the skeleton. As we shall see, this is especially
easy to do for quadrics.

Returning to the surface $Q$ given by $XY-ZW = 0$, we make the important observation that $Q$ is the image of the Segre embedding 
$\psi: \Proj^1 \times \Proj^1 \to \Proj^3$. To see this, note that the embedding gives:
\[   ([t:s] , [u:v] )  \mapsto [tu : tv : su : sv]   \]
Therefore if we let $X$ be the first homogenous coordinate in $\Proj^3$, $Y$ be the fourth, and $Z,W$ be the second and third, we obtain:
\[     XY - ZW = tusv - tvsu = 0   \]
Conversely, for any point $P= [X:Z:W:Y] \in Q$, we want to show that $P \in \imag \psi$. Since $P \in \Proj^3$, at least one of the 
coordinates must be nonzero. Say $X \neq 0$, then the defining equation of $Q$ can be rewritten as $Y = ZW/X$. We take $[t:s] = [X : W]$ 
and $[u:v] = [X : Z]$. Then:
\[ \psi ([t:s] , [u:v] ) = [X^2 : XZ : XW : ZW] = [X : Z : W : ZW/X] = [X:Z:W:Y] \]

The Segre embedding now allows us to parametrize two families of lines in $Q$. Let $[t:s]$ vary over all of $\Proj^1$, and fix $[u:v] =
[\alpha: \beta]$. Then for each value of the parameters $\alpha, \beta$ the image of $\psi$ is a line:
\[     [t:s] \mapsto [\alpha t : \beta t: \alpha s: \beta s]     \]
This is an infinite family of lines $\{L_{\alpha \beta}\}$; actually there is a bijection between lines and points $[\alpha, \beta] \in \Proj^1$.
Moreover, all these lines are disjoint.

We obtain another family of lines $\{M_{\gamma \delta}\}$ by holding $[t:s] = [\gamma : \delta]$ fixed, and letting $[u:v]$ vary over 
$\Proj^1$. Similarly, all lines in this family are disjoint. We want now to look at the intersection $L_{\alpha \beta} \cap M_{\gamma \delta}$:
\[      [\alpha t : \beta t : \alpha s : \beta s] = [\gamma u : \gamma v : \delta u : \delta v]       \]
\[     \Rightarrow \left\{ \begin{array} {c}  {[t:u] = [\gamma : \alpha]} \\ {[t:v] = [\gamma : \beta]} \\ {[s:u] = [\delta : \alpha]} \\ 
	{[s:v] = [\delta : \beta]} \end{array} \right.  \Rightarrow  \left\{ \begin{array} {c}  {[t:s] = [\gamma : \delta]} \\  
	{[u:v] = [\alpha : \beta]} \end{array} \right.  \]
This means there is a unique point of intersection:
 \[ L_{\alpha \beta} \cap M_{\gamma \delta} = [\alpha \gamma : \beta \gamma : \alpha \delta : \beta \delta] = \psi \big( [\gamma : \delta] ,
 [\alpha : \beta] \big)  \]
These two families of lines and the intersections can be seen in figure 2, as the exterior structure of the Sydney tower.

\begin{figure}[h!]
  \caption{Sydney tower}
  \centering
    \includegraphics[width=0.3\textwidth]{tower}
\end{figure}

The previous discussion applies only to nonsingular quadrics. In the case of the singular cone, for example, it's easy to see that
the two families of lines collapse into a single family, and all lines intersect at the vertex of the cone. The reducible quadrics consist of planes,
which contain lines with arbitrary orientations.

\end{document}






























