\documentclass[12 pt]{article}
\usepackage{amsmath,amssymb,amsthm,fullpage,amsfonts,enumerate,textcomp, eurosym}
\title{Representation theory HW1}
\author{Matei Ionita}


\newcommand{\K}{\mathbb{K}}
\newcommand{\R}{\mathbb{R}}
\newcommand{\Q}{\mathbb{Q}}
\newcommand{\Z}{\mathbb{Z}}
\newcommand{\F}{\mathbb{F}}
\newcommand{\C}{\mathbb{C}}
\newcommand{\CP}{\mathbb{C}\mathbb{P}}
\newcommand{\RP}{\mathbb{R}\mathbb{P}}
\newcommand{\Proj}{\mathbb{P}}
\newcommand{\N}{\mathbb{N}}
\newcommand{\p}{\partial}
\newcommand{\fr}{\mathfrak}

\DeclareMathOperator{\Ind}{Ind}
\DeclareMathOperator{\Ker}{Ker}
\DeclareMathOperator{\Tr}{Tr}
\DeclareMathOperator{\Hom}{Hom}
\DeclareMathOperator{\length}{length}
\DeclareMathOperator{\res}{Res}
\DeclareMathOperator{\Int}{Int}
\DeclareMathOperator{\Ext}{Ext}
\DeclareMathOperator{\Aut}{Aut}
\DeclareMathOperator{\Gal}{Gal}
\DeclareMathOperator{\Sym}{Sym}
\DeclareMathOperator{\Lie}{Lie}
\DeclareMathOperator{\id}{Id}
\DeclareMathOperator{\tr}{tr}
\DeclareMathOperator{\irr}{irr}
\DeclareMathOperator{\supp}{supp}
\DeclareMathOperator{\trdeg}{trdeg}
\DeclareMathOperator{\Spec}{Spec}
\DeclareMathOperator{\Nm}{Nm}
\DeclareMathOperator {\HH} {\mathbb{H}}
\DeclareMathOperator {\End} {End}
\DeclareMathOperator {\ad} {ad}
\DeclareMathOperator {\Ad} {Ad}
\DeclareMathOperator {\ch} {char}


\begin{document}
  \maketitle

\subsection*{Problem 1 (8.7 in Kirillov)}

\subsection*{Problem 2 (8.9 in Kirillov)}
$V_n \otimes V_m$ obviously has a decomposition into irreducibles $\bigoplus V_k$; we just need to figure out what the $k$'s are. Using the properties of characters, we get:
\[    \ch (V_n) \ch (V_m) = \sum_k \ch (V_k)    \]
For irreducible reps of $\fr{sl}(2,\C)$, we know that $\ch (V_k) = X^k + X^{k-2} + \dots + X^{-k}$, therefore:
\[           (X^n + X^{n-2} + \dots + X^{-n})(X^m + X^{m-2} + \dots + X^{-m})   =   \sum_k    ( X^k + X^{k-2} + \dots + X^{-k})           \]
After distributing terms, and assuming WLOG that $n\geq m$, the LHS becomes:
\begin{align*}
  \text{LHS} &=   X^{n+m} + 2 X^{n+m-2} + 3 X^{n+m-4} + \dots +  (m+1) X^{n-m} + (m+1) X^{n-m-2} + \dots           \\
 &+ (m+1) X^{-n +m} + m X^{-n +m -2} + \dots + 2 X^{-n-m+2} + X^{-n-m}
\end{align*}
Therefore we identify the $k$'s as $n+m, n+m -2, n+m - 4, \dots, n-m$. In other words, these are the integers that satisfy the Clebsch-Gordan relations. I am a little fatty.




\end{document}




























