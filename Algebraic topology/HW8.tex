\documentclass[12 pt]{article}
\usepackage{amsmath,amssymb,amsthm,fullpage,amsfonts,enumerate,textcomp, eurosym, tikz-cd, fullpage, todonotes}
\title{Algebraic topology HW8}
\author{Matei Ionita}

\newcommand{\R}{\mathbb{R}}
\newcommand{\Q}{\mathbb{Q}}
\newcommand{\Z}{\mathbb{Z}}
\newcommand{\F}{\mathbb{F}}
\newcommand{\C}{\mathbb{C}}
\newcommand{\CP}{\mathbb{C}\mathbb{P}}
\newcommand{\RP}{\mathbb{R}\mathbb{P}}
\newcommand{\Proj}{\mathbb{P}}
\newcommand{\N}{\mathbb{N}}
\newcommand{\p}{\partial}
\newcommand{\fr}{\mathfrak}

\DeclareMathOperator{\Ker}{Ker}
\DeclareMathOperator{\Mor}{Mor}
\DeclareMathOperator{\Hom}{Hom}
\DeclareMathOperator{\length}{length}
\DeclareMathOperator{\res}{Res}
\DeclareMathOperator{\Int}{Int}
\DeclareMathOperator{\Ext}{Ext}
\DeclareMathOperator{\Aut}{Aut}
\DeclareMathOperator{\Gal}{Gal}
\DeclareMathOperator{\Sym}{Sym}
\DeclareMathOperator{\Lie}{Lie}
\DeclareMathOperator{\id}{Id}
\DeclareMathOperator{\tr}{tr}
\DeclareMathOperator{\irr}{irr}
\DeclareMathOperator{\supp}{supp}
\DeclareMathOperator{\trdeg}{trdeg}
\DeclareMathOperator{\Spec}{Spec}
\DeclareMathOperator{\Nm}{Nm}
\DeclareMathOperator{\ord}{ord}
\DeclareMathOperator{\imag}{Im}

\begin{document}
  \maketitle

\subsection*{Problem 1 (Hatcher 2.1.11)}
Let $r : X \to A$ be a retraction, and $\iota : A \to X$ be the inclusion map. Then $r \circ \iota = \id_A$,
which shows $r_* \circ \iota_* = \id_{H_n(A)}$ for all $n$. In particular, the composition $r_* \circ \iota_*$
is bijective, which implies that $\iota_*: H_n(A) \to H_n(X)$ is injective.


\subsection*{Problem 2 (Hatcher 2.1.12)}
Let $A, B$ be chain complexes and let $f, g,h : A_n \to B_n$ be chain maps. By definition, $f$ and $g$
are chain homotopic if there exists a map $P : A_n \to B_{n+1}$ such that $f - g = \p P + P \p$. We
show below that chain homotopy is an equivalence relation.
\begin{enumerate}[(a)]
\item \underline{Reflexivity.} Letting $P = 0$, we have $f - f = \p \circ 0 + 0 \circ \p$.
\item \underline{Symmetry.} Assume $f - g = \p P + P \p$, then $g - f = \p (-P) + (-P) \p$.
\item \underline{Transitivity.} Assume $f - g = \p P + P \p$ and $g-h = \p Q + Q \p$. Then:
	\[	f - h = (f-g) + (g-h) = \p P + P \p + \p Q + Q \p = \p (P + Q) + (P + Q) \p	\]
\end{enumerate}


\subsection*{Problem 3 (Hatcher 2.1.14)}
We construct maps that make the following sequence exact:
\[	0 \to \Z_4 \overset{\phi}{\to} \Z_8 \oplus \Z_2 \overset{\psi}{\to} \Z_4 \to 0	\]
Specifically, define $\phi$ by its action on the generator of $\Z_4$ as $\phi(1) = (2,1)$. The image of
$\phi$ consists of $K = \{(0,0),(2,1),(4,0),(6,1) \}$. To make the sequence exact, we need $K = \Ker \psi$.
This condition makes the image of $\psi$ cyclic, generated by $\psi(1,0)$. Indeed, $\psi(0,1) = \psi(2,1) -
2 \psi(1,0) = - 2 \psi(1,0)$, and we see that $\psi$ is completely determined by its kernel and its action
on the generator:
\[	\psi(m,n) = m \psi(1,0) + n \psi(0,1) = (m-2n) \psi(1,0)	\]
Therefore if we choose $\psi(1,0) = 1$ the image of $\psi$ is cyclic and has order $16/4 = 4$, so $\imag 
\psi \cong \Z_4$ as desired.

To generalize this example, we look for groups $A$ that fit into the exact sequence:
\[	0 \to \Z_{p^m} \overset{\phi}{\to} A \overset{\psi}{\to} \Z_{p^n} \to 0	\]
Observe that $A = \Z_{p^{m+n-k}} \oplus \Z_{p^{k}}$, for $k \leq \min(m,n)$, fit into the sequence. To
see this explicitly, we let the injective map $\phi : \Z_{p^m} \to \Z_{p^{m+n-k}} \oplus \Z_{p^{k}}$
be
\[	1 \mapsto (2^{n-k}, 1)	\]
The image of $\phi$ is $K = (i 2^{n-k}, i)$ for $i \in \Z_{p^m}$, therefore we look for a map $\psi : \Z_{p^{m+n-k}} 
\oplus \Z_{p^{k}} \to \Z_{p^n}$ with kernel $K$. We let $\psi(1,0) = 1$, and we observe that these two properties 
completely determine $\psi$, since:
\[	\psi(i,j) = i\psi(1,0) + j \psi(0,1) = i \psi(1,0) - j \psi(2^{n-k}) = (i - 2^{n-k} j) \psi(1,0) = i - 2^{n-k} j	\]
Note that $K$ is indeed the entire kernel, since $i - 2^{n-k} j = 0$ gives $(i,j) = j (2^{n-k}, 1) \in K$. Finally, $\imag
\psi$ is cyclic generated by $\psi(1,0)$, and has order $p^{n+m} / p^m = p^n$, therefore $\imag \psi \cong \Z_{p^n}$
as desired.

The second type of exact sequence that we need to analyze is:
\[	0 \to \Z \overset{\phi}{\to} A \overset{\psi}{\to} \Z_n \to 0	\]
Observe that $A = \Z \oplus \Z_d$, for $d | n$, fit into the sequence. In this case, we define $\phi$ to be
\[	1 \to (1, n/d)	\]
Therefore we need $\Ker \psi = K = \langle (1, n/d) \rangle$. Setting $\psi(0,1) = 1$ gives:
\[	\psi(i,j) = (j - i n/d) \psi(0,1) = j - i n/d	\]
$K$ is indeed all of the kernel, since $j-in/d = 0$ iff $(i,j) = i (1, n/d)$. Moreover, we showed that $\imag \psi$
is cyclic. We can find the order of $\imag \psi$ by counting the number of lattice points of $\Z \times \Z$ contained
in the parallelogram spanned by $(0,d)$ and $(1, n/d)$. This number is equal to the area of the parallelogram, which
is equal to $d \cdot n/d = n$. Therefore $\imag \psi \cong \Z_n$, as desired.
\\
\\
\\
\\
\\
\\

\subsection*{Problem 4 (Hatcher 2.1.15)}
If $C = 0$, we have the exact sequence:
\[	A \to B \to 0 \to D \to E	\]
Since $\Ker (B \to 0) = B$, exactness at $B$ implies $\imag (A \to B) = B$, i.e. $A\to B$ is surjective.
Next, $\imag (0 \to D) = 0$, so exactness at $D$ implies $\Ker(D \to E) = 0$, i.e. $D\to E$ is injective.

Conversely, assume $A \to B$ is surjective and $D \to E$ is injective. Since $\imag(A \to B) = B$, exactness
at $B$ shows that $\Ker(B \to C) = B$. Since all of $B$ is in the kernel, $\imag (B \to C) = 0$. Exactness at
$C$ then implies that $\Ker(C \to D) = 0$. Moreover, since $D \to E$ is injective, $\Ker (D \to E) = 0$, and
exactness at $D$ gives $\imag (C \to D) = 0$. We see that $C \to D$ has trivial kernel and trivial image,
so $C = 0$.

Finally, we look at the claim about isomorphisms on homology groups. As a consequence of Theorem 2.16
in Hatcher, the following sequence of homology groups is exact:
\[	\dots H_n(A) \to H_n(X) \to H_n(X,A) \to H_{n-1}(A) \to H_{n-1}(X) \dots	\]
Since $H_n(X,A) = 0$, the previous discussion shows that for all $n$ the map $H_n(A) \to H_n(X)$ is
surjective and the map $H_{n-1}(A) \to H_{n-1}(X)$ is injective. Applying this to $n+1$, we get that
the map $H_n(A) \to H_n(X)$ is also injective, and therefore it is bijective.


\subsection*{Problem 5}
\begin{enumerate}[(a)]
\item If $f\circ g$ is chain homotopic to $\id_B$, then by Proposition 2.12 in Hatcher $(f\circ g)_* = 
\id_{H_n(B)}$ for every $n$. In particular, $(f\circ g)_* = f_* \circ g_*$ is bijective, hence $f_*$ is
surjective and $g_*$ is injective. Similarly, $g \circ f$ chain homotopic to $\id_A$ implies that $g_*$ is
surjective and $f_*$ is injective. Putting these together, we get that $f_* : H_n(A) \to H_n(B)$ is a bijective
homomorphism, i.e. an isomorphism.
\item Consider the chain complexes $A$ and $B$ pictured below:
\[
\begin{tikzcd}
\cdots\arrow{r}&0\arrow{r}&\Z_2\arrow{r}{\iota}\arrow{d}{f=0}&\Z_4\arrow{r}{q}\arrow{d}{f=0}&
\Z_2\arrow{r}\arrow{d}{f=0}&0\arrow{r}&\cdots \\
\cdots\arrow{r} & 0\arrow{r} & \Z\arrow{r}{\id}\arrow{d}{g} & \Z\arrow{r}\arrow{d}{g} & 
0\arrow{r}\arrow{d}{g} & 0\arrow{r} & \cdots \\
\cdots\arrow{r}&0\arrow{r}&\Z_2\arrow{r}{\iota}&\Z_4\arrow{r}{q}&\Z_2\arrow{r}&0\arrow{r}&\cdots
\end{tikzcd}
\]
The first and third row both represent the complex $A$, while the middle row is complex $B$. Since both sequences 
are exact, both complexes have trivial homology. Moreover, any chain map $f : A \to B$ must be zero in all
gradings, because there exists no nonzero homomorphism $\Z_n \to \Z$. Therefore for any chain map $g:
B \to A$, $g\circ f = 0$ in all gradings. Chain homotopy equivalence between $A$ and $B$ implies, then, the
existence of homomorphisms $P_n : A_n \to A_{n+1}$ such that $\id_{A_n} - 0 = \p P_{n+1} + P_n \p$. We show
that no such homomorphisms can exist in the following parallelogram:
\[
\begin{tikzcd}
{} & \Z_4\arrow{r}{q}\arrow[swap]{ld}{P_{n+1}} & \Z_2\arrow{ld}{P_n} \\
\Z_2\arrow{r}{\iota} & \Z_4 & 
\end{tikzcd}
\]
The maps $\iota$ and $q$ are fixed as the inclusion and quotient map respectively. $P_{n+1}$ can be either $q$
or the zero map, since these are the only two homomorphisms $\Z_4 \to \Z_2$. Similarly, $P_n$ can be either
$\iota$ or the zero map. We analyze what happens when both of them are nonzero:
\begin{align*}
(\p P_{n+1} + P_n \p) (1) &= (\iota \circ q + \iota \circ q) (1) \\
&=  \iota (q(1)) + \iota (q(1)) \\
&= \iota(1) + \iota(1) \\
&= 2 + 2 \\
&= 0
\end{align*}
Thus $\p P_{n+1} + P_n \p$ cannot be equal to the identity map on $\Z_4$. If we choose instead to make either
$P_n$ or $P_{n+1}$ the zero map, then one of the two terms drops, and we obtain $(\p P_{n+1} + P_n \p) (1)
= 2$, which is still not the identity map. Finally, if we make both $P_n$ and $P_{n+1}$ the zero map, it's clear
that we don't get the identity on $\Z_4$. Therefore the two complexes are not chain homotopy equivalent.
\end{enumerate}



\end{document}


















