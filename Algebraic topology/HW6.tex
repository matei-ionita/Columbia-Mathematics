\documentclass[12 pt]{article}
\usepackage{amsmath,amssymb,amsthm,fullpage,amsfonts,enumerate,textcomp, eurosym, tikz-cd, fullpage, todonotes}
\title{Algebraic topology HW6}
\author{Matei Ionita}

\newcommand{\R}{\mathbb{R}}
\newcommand{\Q}{\mathbb{Q}}
\newcommand{\Z}{\mathbb{Z}}
\newcommand{\F}{\mathbb{F}}
\newcommand{\C}{\mathbb{C}}
\newcommand{\CP}{\mathbb{C}\mathbb{P}}
\newcommand{\RP}{\mathbb{R}\mathbb{P}}
\newcommand{\Proj}{\mathbb{P}}
\newcommand{\N}{\mathbb{N}}
\newcommand{\p}{\partial}
\newcommand{\fr}{\mathfrak}

\DeclareMathOperator{\Mor}{Mor}
\DeclareMathOperator{\Hom}{Hom}
\DeclareMathOperator{\length}{length}
\DeclareMathOperator{\res}{Res}
\DeclareMathOperator{\Int}{Int}
\DeclareMathOperator{\Ext}{Ext}
\DeclareMathOperator{\Aut}{Aut}
\DeclareMathOperator{\Gal}{Gal}
\DeclareMathOperator{\Sym}{Sym}
\DeclareMathOperator{\Lie}{Lie}
\DeclareMathOperator{\id}{Id}
\DeclareMathOperator{\tr}{tr}
\DeclareMathOperator{\irr}{irr}
\DeclareMathOperator{\supp}{supp}
\DeclareMathOperator{\trdeg}{trdeg}
\DeclareMathOperator{\Spec}{Spec}
\DeclareMathOperator{\Nm}{Nm}
\DeclareMathOperator{\ord}{ord}
\DeclareMathOperator{\imag}{Im}

\begin{document}
  \maketitle

\subsection*{Problem 1 (1.3.10 in Hatcher)}
Since the covering map is a local homeomorphism, any 2-sheet covering of $S^1 \vee S^1$ must be a graph with two 
vertices, each of degree 4. Specifically, each vertex must have an $a$ and a $b$ edge coming out of it, and an $a$ 
and a $b$ edge going into it. There are only three such graphs, pictured below.
\\
\\
\\
\\
\\
\\
\\
\\
\\
\\
\\
Isomorphism classes of covering spaces without regard to basepoint are in 1-1 correspondence with conjugacy 
classes of subgroups of $\Z * \Z$. Since our covers have 2 sheets, their fundamental groups have index 2 in 
$\Z * \Z$. All index 2 subgroups are normal, which means that each is its own conjugacy class. Therefore none 
of the three covers presented above are isomorphic.

For 3-sheet covers, we look for graphs with 3 vertices, each with degree 4. We show these below, in increasing 
order of the number of self-loops.
\\
\\
\\
\\
\\
\\
\\
\\
\\
\\
\\
\\
\\
\\
\\
\\
\\
It's straightforward to check that none of the subgroups associated to these 7 graphs are conjugates of each
other; in fact, the first two and last two are normal subgroups. Therefore they all represent different isomorphism
classes of covering spaces.


\subsection*{Problem 2 (1.3.18 in Hatcher)}
Consider all abelian covering spaces $\tilde X_i$ of $X$, and denote their deck groups by $G_i$. We identify
$\pi_1(\tilde X_i)$ with its image by the injective map $p_{i*}$, and the result of Proposition 1.39 becomes
\[      G_i \cong \pi_1(X) / \pi_1(\tilde X_i)     \]
Since $\pi_1(\tilde X_i)$ are normal subgroups of $\pi_1(X)$, so is their intersection $N= \bigcap_i \pi_1(\tilde X_i)$.
Via Theorem 1.38, $N$ is the fundamental group of a covering space $Y$ of $X$, which is normal because $N$
is. We want to show that $G = \pi_1(X) /N$, the deck group of $Y$, is abelian. Recall that a quotient by a 
normal subgroup is abelian iff the subgroup contains all elements of the form $aba^{-1}b^{-1}$. We know that $G_i$
are abelian, so these elements are in each $\pi_1(\tilde X_i)$. Therefore they are also in the intersection $N$, which
proves that $Y$ is an abelian covering space. Moreover, by construction $N = \pi_1(Y)$ is a subgroup of 
$\pi_1(\tilde X_i)$ for every $i$. Using again the correspondence of Theorem 1.38, $Y$ is a covering space for
all $\tilde X_i$.

Now assume there are two universal abelian coverings, $Y$ and $Z$. Then $Y$ covers $Z$, so $\pi_1(Y)$ is isomorphic
to a subgroup of $\pi_1(Z)$; the converse is analogous. Therefore $\pi_1(Y) \cong \pi_1(Z)$. But $Y$ and $Z$ are
covering spaces of $X$, and in particular their fundamental groups are isomorphic as subgroups of $\pi_1(X)$.
Then by Proposition 1.37 $Y$ and $Z$ are isomorphic covering spaces.

For the case $X = S^1 \vee S^1$, we have $\pi_1(X) = \Z * \Z$. The deck group of the universal abelian cover
is the maximal abelian quotient of $\Z * \Z$. It's clear that to obtain an abelian quotient we must set
$aba^{-1}b^{-1} = 1$; if we want the quotient to be maximal, we impose nothing else. Therefore the deck group
of the universal abelian cover is $\Z \times \Z$. Motivated by this, we consider the grid $Y$ formed from the lattice
$\Z^2 \subset \R^2$ by connecting lattice points with vertical and horizontal lines. The quotient map $p:Y \to X$
that takes $(x,y) \mapsto (x \mod 1, y \mod 1)$ is easily seen to be a covering map, and $G(Y) = \Z \times \Z$ as
desired. Finally, $X = S^1 \vee S^1 \vee S^1$ gives the same result in 3 dimensions, i.e. $Y$ is the 3-dimensional
grid obtained from $\Z^3\subset \R^3$, and $G(Y) = \Z \times \Z \times \Z$.



\subsection*{Problem 3}
This solution is inspired by Example 1.41 in Hatcher. Let $X = S^1 \vee S^1$ and let $Y = \bigvee_{n} S^1$. We 
can view $Y$ as ``keys on a keychain'' as in the picture, by keeping one circle fixed and translating the points of
 tangency of the other $n-1$ circles. This operation is a homotopy equivalence, so we just take this resulting 
space to be $Y$, and $\pi_1(Y)$ will still be the free group on $n$ generators.
\\
\\
\\
\\
\\
\\
\\
\\
\\
\\
$G = \Z/n-1$ acts freely on $Y$ in the obvious way (the generator rotates the whole figure by $2\pi/{n-1}$ 
radians). Since $G$ is finite and the action is free, it is a covering space action. (See comments on p. 73 in 
Hatcher.) The quotient $Y/G \cong X$, and by Proposition 1.40 $Y$ is a covering space of $X$. This means 
that $p_*(\pi_1(Y))$ is a subgroup of $\pi_1(X)$. Moreover, $p_*$ is injective, so this gives an injection of 
the free group on $n$ generators into the free group on 2 generators.

\subsection*{Problem 4}
The fundamental group of the Klein bottle is
\[      G =    \langle a,b | abab^{-1} \rangle        \]
Note that $abab^{-1} = 1$ implies that $a^kba^kb^{-1} = 1$ for all natural $k$. This can be proved by
induction, using
\[         a^k b a^k b^{-1} = a^{k-1}( abab^{-1}) ba^{k-1} b^{-1} = a^{k-1}ba^{k-1}b^{-1}          \]
Consider the injective homomorphism $\phi_k : \langle a^k, b \rangle \to \langle a, b \rangle$; because of
the remark made above, it descends to an injective homomorphism on the quotients
\[         \psi_k   :  H_k = \langle a^k, b | a^kba^kb^{-1} \rangle \to G     \]
Each $H_k$ is the fundamental group of a $k$-sheet covering of the Klein bottle by itself. The covering for 
$k=3$ is pictured below, and the others are obtained similarly.
\\
\\
\\
\\
\\
\\
\\
It suffices, then, to find a value of $k$ such that $H_k$ is not normal in $G$. $k=2$ will not work, as any
index two subgroup is normal. We try the next simplest case, which is $k=3$. Conjugating $b \in H_3$ by
$a \in G$ gives
\[       aba^{-1} = (abab^{-1}) b a^{-2} = b a^{-2}  = (ba^{3}) a^{-1}  \]
Since $ba^3 \in H_3$, $aba^{-1} \in H_3$ would imply $a\in H_3$, which is a contradiction. This shows
that $H_3$ is not normal, therefore the covering space pictured above is not normal.

To get a covering by the torus, we can first look at the 2-sheet covering given by $\langle a, b^2 \rangle$.
Note that the relation $abab^{-1} = 1$ in $G$ implies that $a$ and $b^2$ commute:
\[           ab^2 = ba^{-1}b = b^{2}a (a^{-1}b^{-1}a^{-1}b) = b^2a          \]
Therefore this is a $\Z^2$ subgroup of $G$. Any 2-generator subgroup of this subgroup will itself give a
covering of the Klein bottle by a torus, because it will be a 2-generator abelian group, which is the 
fundamental group of the torus. Therefore it suffices to find a subgroup of $\langle a,b^2|ab^2 = b^2a
\rangle$ that is not normal in $G$. We try:
\[   H =   \langle a^3, a^2b^2 | abab^{-1} \rangle     \]
Conjugating $a^2b^2$ by $b$ gives:
\[        b(a^2b^2)b^{-1} = bab^{-1}bab^{-1}b^2 = a^{-2}b^2 = a^{-4} (a^2b^2)     \]
If this is an element of $H$, then so must be $a^{-4}$, which is not the case. Therefore conjugating by
$b$ produces an element that is not in $H$, so $H$ is not normal. $H$ determines a 6-sheet covering
of the Klein bottle by a torus, which is pictured below.
\\
\\
\\
\\
\\
\\
\\
\\
\\

\subsection*{Problem 5}
\begin{enumerate}[(a)]
\item Consider a space $Y$ that covers $X/G$ and is covered by $X$:
	\[        X \to Y \to X/G     \]
	This determines a chain of inclusions of fundamental groups:
	\[         \pi_1(X) \leq \pi_1(Y) \leq \pi_1(X/G)           \]
	$\pi_1(X)$ is normal in $\pi_1(X/G)$, because $X$ is a normal covering space. But, if every 
	$g \in \pi_1(X/G)$ normalizes $\pi_1(X)$, in particular every $g\in \pi_1(Y)$ will do so. Therefore 
	$\pi_1(X)$ is normal in $\pi_1(X/G)$, and then it makes sense to quotient the chain of inclusions:
	\[        1 \leq \frac{\pi_1(Y)}{ \pi_1(X)} \leq \frac{\pi_1(X/G) }{ \pi_1(X)  } = G  \]
	Let $H = \pi_1(Y) / \pi_1(X)$, we have just shown that $H \leq G$. Since $\pi_1(X)$ is normal in
	$\pi_1(Y)$, $Y \to X$ is a normal covering space. From Proposition 1.39 it follows that $H$ is the
	deck group of the covering, hence $Y = X/H$.
\item Assume first that $H_2 = gH_1g^{-1}$ for some $g\in G$, then $H_1 \cong H_2$ via 
	$h \mapsto ghg^{-1}$. Using the definition of $H$ from part (a), we get
	\[    \frac{  \pi_1(Y_1) }{\pi_1(X)} \cong \frac{\pi_1(Y_2)}{\pi_1(X)}    \]
	By results of group theory, this implies that $\pi_1(Y_1)$ and $\pi_1(Y_2)$ are conjugate subgroups
	of $\pi_1(X/G)$. Then the correspondence in Theorem 1.38 shows that $Y_1$ and $Y_2$ are
	isomorphic covering spaces.

	Conversely, assume that $X/H_1$ and $X/H_2$ are isomorphic as covering spaces of $X/G$. By
	Theorem 1.38 $\pi_1(X/H_1)$ and $ \pi_1(X/H_2)$ are conjugate subgroups of $\pi_1(X/G)$. Then
	their quotients by their common normal subgroup $\pi_1(X)$ are conjugate subgroups of
	$\pi_1(X/G) / \pi_1(X) = G$.
\item Suppose that $X/H \to X/G$ is normal, then $\pi_1(X/H) \unlhd \pi_1(X/G)$. We apply the third
	isomorphism theorem, which says precisely that $H = \pi_1(X/H) / \pi_1(X)$ is normal in 
	$G = \pi_1(X/G)/\pi_1(X)$, and that:
	\[      \frac{  \pi_1(X/G)}{\pi_1(X/H)} = \frac{ \pi_1(X/G)/\pi_1(X)}{\pi_1(X/H) / \pi_1(X)} = G/H        \]
	By Proposition 1.39, the LHS is equal to the deck group of $X/H \to X/G$, which gives the desired result.

	Conversely, assume that $H \unlhd G$, which is to say
	\[       \frac{ \pi_1(X/H)}{\pi_1(X)} \unlhd \frac{\pi_1(X/G) }{\pi_1(X)}      \]
	It follows from basic group theory that $\pi_1(X/H) \unlhd \pi_1(X/G)$, which means that $X/G \to X/H$
	is normal.
\end{enumerate}



\end{document}


























