\documentclass[12 pt]{article}
\usepackage{amsmath,amssymb,amsthm,fullpage,amsfonts,enumerate,textcomp, eurosym}
\title{QFT Lecture 17}
\author{Matei Ionita}
\DeclareMathOperator {\p} {\partial}

\begin{document}
  \maketitle

\subsection*{Continuous symmetries}
Reference: Srednicki 22
\\
\\
For a symmetry that is path connected to the identity, we know: Classically, Noether's theorem, $\p_{\mu} j^{\mu} = 0$ on the e.o.m. Quantum, Ward identity:
\[     \p_{\mu} \langle j^{\mu}(x_0) \mathcal{O}(x_1)\rangle   = \delta(x_0 - x_1) \langle \delta \mathcal{O} (x_1)\rangle   \]
From this we can deduce:
\[   [ \hat Q , \mathcal{O} ] = \delta \mathcal{O}   \]
Classically, we have $\frac{\delta S}{\delta \phi} = 0$ on the e.o.m. In quantum, $\langle \frac{\delta S}{\delta \phi (x_0)} , \phi(x_1) \rangle = ?$ To compute this, first note that, by a twisted analog of the fundamental theorem of calculus:
\[       \int D\phi \frac{\delta (\phi(x_1) e^{iS})}{\delta \phi(x_0) }   = 0    \]
Where we have used the fact that the fields decay to 0 at $\infty$. Schwinger - Dyson eq.:
\[        \left\langle  \frac{\delta S}{\delta \phi (x_0)} \phi(x_1) ... \phi(x_n) \right\rangle   = i \delta (x_1 - x_0) \langle \phi (x_2) ... \phi (x_n) \rangle +...     \]
\\
\\
Continuous symmetries of two types:
\\
a) internal. ex. $\phi \to \phi + \text{const}$
\\
b) spacetime, ex. translations $\delta \phi = a^{\mu} \delta_{\mu} \phi$

\subsection*{Discrete symmetries}
Reference: Srednicki 23
\\
\\
Look at $\phi \to -\phi$. As a consequence, if we have an even number of incoming particles, we should have an even number of outgoing.
\\
\\
Let $\hat Z$ be the operator that generates the symmetry transformation: $\hat Z^{-1} \hat \phi \hat Z = - \hat \phi$. Let's suppose our theory is symmetric, i.e. $[\hat Z, \hat H] = 0$. Also suppose $\hat Z |0\rangle = |0\rangle$. Then $\langle 0| \hat \phi |0 \rangle = 0$.
\\
\\
Recall that for a Lorentz transformation:
\[     U^{-1} (\Lambda) \hat \phi(x) U(\Lambda) = \hat \phi(\Lambda^{-1} x)    \]
Parity:
\[     U^{-1} (\mathbb{P}) \hat \phi(x) U(\mathbb{P}) = \hat \phi(\mathbb{P}^{-1} x)    \]
\[     \mathbb{P} = \left(  \begin{array} {cccc}   1 & & & \\ & -1 & & \\ & & -1 & \\ & & & -1 \\   \end{array}  \right)          \]
Let $P = U(\mathbb{P})$, and define a pseudoscalar as: $P^{-1} \phi (x) P = - \phi(\mathbb{P}^{-1} x)$. For interactions with only even powers of $\phi$, it doesn't matter if $\phi$ is scalar or pseudoscalar, but if we have odd terms, need it to be scalar to have symmetry.














































\end{document}