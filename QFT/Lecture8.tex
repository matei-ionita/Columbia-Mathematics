\documentclass[12 pt]{article}
\usepackage{amsmath,amssymb,amsthm,fullpage,amsfonts,enumerate,textcomp, eurosym}
\title{QFT Lecture 8}
\author{Matei Ionita}

\begin{document}
  \maketitle

\subsection*{How to think about the scalar field?}

Let's look for a second about an ocean wave. The e.o.m. is:
\[ \left(  - \frac{\partial^2}{\partial t^2} + c_0^2 \frac{\partial^2}{\partial x^2} \right) \phi(\mathbf{x}, t) \]
The action, which is NOT Lorentz invariant:
\[  S = \int dt d^3x \left(  \frac{1}{2} (\partial_t \phi)^2 - \frac{c_0^2}{2} (\partial_x \phi)^2  \right)  \]
This wave does not have an abvious particle interpretation. In the quantum case, it's important that we cannot have a simply flat wave, not even in the vacuum state:
\[  \langle 0 | T \hat \phi_1 \hat \phi_2 | 0 \rangle \]
Eigenvalues and eigenstates of the field:
\[  \hat \phi (x) |f(x)\rangle = f(x) |f(x)\rangle  \]
ANY 4D configuration is an eigenstate of the field, as long as it evolves according to the Klein-Gordon eq. Notice that it evolves backward in time:
\[  \hat \phi (\mathbf{x}, t) = e^{iHt} \hat\phi (\mathbf{x}, 0) e^{-iHt}  \]
\[  \hat \phi (\mathbf{x}, t) |f (\mathbf{x}), t\rangle = f (\mathbf{x}) |f (\mathbf{x}), t\rangle  \]
\[  e^{iHt} \hat\phi (\mathbf{x}, 0) e^{-iHt} |f (\mathbf{x}), t\rangle = f (\mathbf{x}) |f (\mathbf{x}), t\rangle  \]
\[  \hat \phi (\mathbf{x}, 0) |f (\mathbf{x}), 0\rangle = f (\mathbf{x}) |f (\mathbf{x}),0\rangle  \]
So:
\[  |f (\mathbf{x}), t\rangle = e^{iHt} |f (\mathbf{x}), 0\rangle  \]

\subsection*{Bubble diagrams}

\[   \mathcal{L}_{int} = - \frac{1}{2} (Z_{\phi} - 1) (\partial \phi)^2 - \frac{1}{2} (Z_{m} - 1) m^2 \phi^2 + \frac{1}{3!}Z_g g \phi^3 + Y\phi + \Lambda \]
$\Lambda$ is related to the cosmological constant. Somehow, quantum mechanics forces all these counter-terms on us whenever we try to introduce the $\phi^3$ interaction. Last time we studied:
\[  Z_1(J) = \int D\phi e^{i\int d^4 x (\mathcal{L}_0 + \frac{1}{3!} Z_g g + J\phi)}  \]
There's a theorem saying that:
\[   Z_1(J) = e^{\sum connected\;diagrams}  \]
To prove it, consider the sum:
\[  \sum_{\{n_I\}} D^{\{n_I\}}  \]
The collection of $n_I$'s shows how many of each connected bubbles are in D, so $D = \Pi_I (C_I)^{n_I}$. Introduce a symmetry factor $s_D = \Pi_I (n_I !)$.
\[  \sum_{\{n_I\}} D^{\{n_I\}} =  \sum_{\{n_I\}} \Pi_I \frac{C_I^{n_I}}{n_I!} = \Pi_I \sum_{\{n_I\}} \frac{C_I^{n_I}}{n_I!} = \Pi_I e^{C_I} = e^{\sum C_I} \]
Think about why we can interchange the sum and product!





\end{document}