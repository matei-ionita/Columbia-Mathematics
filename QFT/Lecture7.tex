\documentclass[12 pt]{article}
\usepackage{amsmath,amssymb,amsthm,fullpage,amsfonts,enumerate,textcomp, eurosym}
\title{QFT Lecture 7}
\author{Matei Ionita}

\begin{document}
  \maketitle

\section*{Field theory with interactions}
Use perturbation theory! This will eventually lead to Feynman diagrams. Zee I-7, I-11. Sredniki 9.
\\
\\
Recall the LSZ formula; it tells us that the scattering amplitude is related to n-point functions. Goal: compute n-point function for any theory, in which the action is not quadratic in the field.
\[  \langle \phi_1 \phi_2 ... \phi_n\rangle = \int D\phi e^{iS} \phi_1 ... \phi_n  \]
Example:
\[ S = \int d^4 x (\mathcal{L}_0 + \mathcal{L}_1 )  \]
\[  \mathcal{L}_0 = -\frac{1}{2} (\partial \phi)^2 - \frac{1}{2} m^2 \phi^2 \;\;\;\; \mathcal{L}_1 = \frac{1}{6} g \phi^3 \]
$\mathcal{L}_1$ is a toy Lagrangian that we will be exploring the next weeks. Note that g represents the self-coupling of the field, and not its coupling with something else.

\subsection*{Perturbation theory}
We compute $ \langle \phi_1 \phi_2 ... \phi_n\rangle$ perturbatively in g.
\[   \langle \phi_1 \phi_2 ... \phi_n\rangle = \int D\phi e^{iS_{free}} e^{i\int d^4 x \frac{1}{6} g \phi^3 (x)}  \phi_1 \phi_2 ... \phi_n \]
\[ e^{i\int d^4 x \frac{1}{6} g \phi^3 (x)} = 1 + i\int d^4 x \frac{g}{6} \phi^3 (x) + ...  \]
\[  \langle \phi_1 ... \phi_n\rangle = \langle \phi_1 ... \phi_n\rangle_{free}   +  \int D\phi e^{iS_{free}} \frac{ig}{6} \int d^4 x \;\phi^3(x)  \phi_1 ... \phi_n + ...  \]
The first-order term is just:
\[ \frac{ig}{6} \int d^4 x  \langle\phi^3(x) \phi_1 ... \phi_n\rangle_{free}  \]
Note that if n is even, the n+3 point function is 0, so we mst go to second order in g to get nontrivial scattering. For $n=4$, i.e. 2 - 2 scattering, we have:
\[  \langle \phi_1 \phi_2 \phi_3 \phi_4 \rangle = \langle \phi_1 ... \phi_4 \rangle_{free} +  \left(\frac{ig}{6}\right)^2 \frac{1}{2} \int d^4 x d^4 y \langle \phi^3(x) \phi^3(y) \phi_1 ... \phi_4 \rangle_{free} \]
In the 10 point function, we have 945 terms. Shit! One of them looks like:
\[  \langle \phi_x \phi_1 \rangle \langle \phi_x \phi_2 \rangle \langle \phi_y \phi_3 \rangle \langle \phi_y \phi_4 \rangle \langle \phi_x \phi_4 \rangle  \]
Represent this in the Feynman diagram (see notebook). Every line uniting two vertices represents a 2-point function; the $\phi_x \phi_y$ term coresponds to the $\phi$ - $\phi$ exchange that mediates the interaction.

\subsection*{More abstract approach (Sredniki 9)}
\[  S_J =  \int d^4 x (\mathcal{L}_0 + \mathcal{L}_1 + J\phi) \]
\[  \mathcal{L}_0 + \mathcal{L}_1 = - \frac{1}{2} Z_{\phi} (\partial \phi)^2 - \frac{1}{2} Z_m m^2 \phi^2 + \frac{1}{2} Z_g g \phi^3 + Y \phi \]
The Z and Y constants are included here for the purpose of renormalization (we'll see later). Note that the Y term and the J one are different, since J is a function of x and Y is a constant. We want to impose:
\\
i) m is the physical mass (the rest mass that we measure in the lab); this imposes choosing some $Z_m$. this is called mass renormalization
\\
ii) g is the physical coupling; this imposes the coupling renormalization
\\
iii) $\langle 0| \phi(x) |0\rangle = 0$. this imposes the linear renormalization
\\
iv) $\langle k| \phi(x) |0\rangle = e^{-ikx}$. this imposes $Z_{\phi}$. for historical reasons, this is called wavefunction normalization.
\\ 
Actually, the last two conditions were implicitly assumed in the derivation of the LSZ formula. We also want to keep the previous definition of $\mathcal{L}_0$:
\[   \mathcal{L}_0 = -\frac{1}{2} (\partial \phi)^2 - \frac{1}{2} m^2 \phi^2   \]
Therefore:
\[ \mathcal{L}_1 = - \frac{1}{2} (Z_{\phi} - 1) (\partial \phi)^2 - \frac{1}{2} (Z_{m} - 1) m^2 \phi^2 + \frac{1}{6} Z_g g \phi^3 + Y \phi \]
For now, let's focus on the g term. The generating function:
\[  Z_1 (J) = \int D\phi e^{i\int d^4 x \; (\mathcal{L}_0 + \frac{1}{6} Z_g g \phi^3 + J\phi)}  \]
Recall the generating function for the free theory:
\[  Z_0 (J) = \int D\phi e^{i\int d^4 x\; (\mathcal{L}_0 + J\phi)}   \]
Claim:
\[  Z_1 (J) =   e^{i \int d^4 x \frac{Z_g g}{6} ( \frac{\delta}{i \delta J(x)})^3  }  Z_0 (J)  \]
We rewrite:
\[ Z_0 (J) = \int D\phi e^{\frac{i}{2} \int d^4 z d^4 y J_y \Delta_{yz} J_z}  \]
Expanding both exponentials:
\[  Z_1 (J) = \sum_{v=0}^{\infty} \frac{1}{v!} \left[ \frac{iZ_g g}{6} \int d^4 x (\frac{\delta}{i \delta J_x})^3  \right]^v  \sum_{P=0}^{\infty} \frac{1}{P!} \left[  \frac{1}{2} \int d^4 y d^4 z (iJ_y) \frac{1}{i} \Delta_{yz} (iJ_z)  \right]^P \]
Let's organize the terms by the no. of J's left $=$ E (external)
\[  E = 2P - 3v  \]
Example for $E=0$ : $V=2, P=3$.
\[  \frac{1}{2!} \left[ \frac{iZ_g g}{6} \int d^4 x (\frac{\delta}{i\delta J_x})^3 \right] \left[ \frac{iZ_g g}{6} \int d^4 y (\frac{\delta}{i\delta J_y})^3 \right]  \frac{1}{3!} \left[ \frac{1}{2} \int d^4a d^4a' (iJ_a) \frac{1}{i} \Delta_{aa'} (iJ_a')  \right] .... \]
\[ = (iZ_g g)^2 \int d^4 x d^4 y (\frac{1}{i} \Delta_{xy})^3  \]



\end{document}



















