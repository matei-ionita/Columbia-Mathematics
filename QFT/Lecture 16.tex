\documentclass[12 pt]{article}
\usepackage{amsmath,amssymb,amsthm,fullpage,amsfonts,enumerate,textcomp, eurosym}
\title{QFT Lecture 16}
\author{Matei Ionita}
\DeclareMathOperator {\p} {\partial}

\begin{document}
  \maketitle

\subsection*{Quantum action}

To find the relation between $W[J]$ and $\Gamma(\phi)$ we define:
\[      e^{iW_{\Gamma}[J]}   = \int D\phi e^{i(\Gamma(\phi) + \int d^d x J\phi)}    \]
Note that $W_{\Gamma}$ at three level corresponds to $W$ with all loops. At the stationary point $\phi_J$:
\[        \int D\phi e^{i(\Gamma(\phi_J) + \int d^d x J\phi_J)}  =     \int D\phi e^{i(\Gamma(\phi_J + \delta \phi) + \int d^d x J( \phi_J + \delta \phi))}      \]
This gives the "quantum equation of motion":
\[   \left.    \frac{\delta \Gamma(\phi)}{\delta \phi(x)}  + J(x)\right|_{\phi(x) = \phi_J} = 0    \]
Close to $\phi_J$:
\[      \int D\phi e^{i(\Gamma(\phi) + \int d^d x J\phi)}  =   e^{i(\Gamma(\phi_J) + \int d^d x J\phi_J)}  \int D(\delta \phi) e^{i/2 \int d^d x_1 d^d x_2 \left. \frac{\delta^2 \Gamma}{\delta \phi_1 \delta \phi_2} \right|_{\phi = \phi_J} \delta \phi_1 \delta \phi_2}  \]
Therefore:
\[    W[J] \text{  (to all orders)}= \Gamma(\phi_J) + \int d^d x J(x) \phi_J(x)         \]
Why is the quantum action useful? We use it to study spontaneous symmetry breaking. Note that $\Gamma$ contains the usual derivative terms (such as $-1/2 (\p \phi)^2$), then some other terms which we collectively call "quantum potential". This is just some sum of quadratic, cubic, etc terms. For example, if we have:
\[     U(\phi) = \frac{1}{2} \alpha \phi^2 + \lambda \phi^4       \]
for negative $\alpha$ and positive $\lambda$ we get a double-well potential. For the usual quadratic potential, the well represents the vacuum expectation value of the field (i.e. 0). For the double-well, the ground state will not be 0 anymore, thus the state is not symmetric, even though the action looks symmetric.


\subsection*{Continuous symmetries}

Later we'll go back to renormalization and think about it in the context of symmetry. For now, reference chapter 22.
\\
\\
Recall that continuous symmetries are the ones that can be smoothly deformed to 1. Classically we have Noether's theorem: for every symmetry there is a conserved current; today we make a quantum statement. Recall the proof of Noether's theorem, which we will use in a bit to prove the quantum statement. S is invariant under $\phi \to \phi+\delta \phi$. We consider $\phi \to \phi + \rho \delta \phi $, where $\rho$ is some arbitrary function. If $\rho$ is a constant, $\delta S$ is 0, so $\delta S$ can only depend on derivatives of $\rho$.
\[      \delta S = \int d^d x j^{\mu} \p_{\mu} \rho        \]
This is a general statement, for any field configuration; it does not assume $\phi$ obeys the classical EOM. We impose this as $\delta S = 0$ for fluctuations around this classical field. Therefore:
\[        \int d^d x j^{\mu} \p_{\mu} \rho   = 0 =  -  \int d^d x  \p_{\mu} j^{\mu}  \rho   \]
using integration by parts. Since this must vanish for every $\rho$,we get $\p_{\mu} j^{\mu} = 0$. This current is 0 AS LONG AS $\phi$ is on-shell.
\\
\\
For the quantum case, consider
\[  \langle \phi(x_1) ... \phi(x_n) \rangle = \int D\phi \phi_1 ... \phi_n e^{iS[\phi]}   \]
Do the change of variables $\phi' = \phi + \delta \phi \rho $.
\[       \langle \phi(x_1) ... \phi(x_n) \rangle = \int D(\phi + \delta\phi \rho) (\phi_1 + \delta \phi \rho) ... (\phi_n + \delta \phi \rho) e^{i(S[\phi]+ \int d^d x j^{\mu} \p_{\mu} \rho)}           \]
For now, we only consider symmetries for which the measure remains unchanged, so $D(\phi + \delta\phi \rho) = D\phi$. The 0 order terms are just the n-point function we started with. Therefore the other terms have to be 0. We'll write the first order terms and ignore the other:
\[        \int D\phi \left[    \rho_1 \delta \phi_1 \phi_2 ... \phi_n  + ... + \phi_1 ... \phi_{n-1} \rho_n \delta \phi_n  + \phi_1 ... \phi_n i \int d^d x j^{\mu} \p_{\mu} \rho   \right]   e^{iS[\phi]}    \]
Since $\rho$ is arbitrary, we can choose it to be $\rho(x) = \delta (x-x_0)$ for some $x_0$. We get:
\[            \int D \phi \phi_1 ... \phi_n (- i \p_{\mu} j^{\mu} )_{x_0} e^{iS[\phi]}  +     \int D\phi \delta(x_1 - x_0) \delta \phi_1 \phi_2 ... \phi_n + ...          \]
\[       i \langle \p_{\mu} j^{\mu}(x_0) \phi_1 ... \phi_n \rangle    =  \delta (x_1 - x_0)  \langle (\delta \phi_1) \phi_2 ... \phi_n \rangle  + ... + \delta (x_n - x_0) \langle \phi_1 ... \phi_{n-1} (\delta \phi_n ) \rangle    \]
The terms on the right are called contact terms. The expectation value is nonzero if the divergence of j happens to be evaluated at one of the field points. This quantum version of Noether's theorem is called the Ward identity. In a simpler and more general form:
\[      i \langle \p_{\mu} j^{\mu} \mathcal{O}(\phi(x_1)) \rangle =    \delta (x_0 - x_1)  \langle \delta \mathcal{O}_1 \rangle     \]
Recall the classically conserved charge:
\[        Q = \int d^{d-1} j^0     \]
In the quantum case:
\[        \int d^{d-1} x     i \langle \p_{\mu} j^{\mu} \mathcal{O}(\phi(x_1)) \rangle  =  \delta(t-t_1) \langle \delta \mathcal{O}_1 \rangle          \]
Using Stokes and throwing away the boundary terms on the LHS, this becomes:
\[          i \p_t \langle Q(t) \mathcal{O} (t_1)    \rangle = \delta(t-t_1) \langle \delta \mathcal{O}_1\rangle     \]
Integrating wrt time from $t_1 - \epsilon$ and $t_1 + \epsilon$ on both sides:
\[     i \langle Q(t+\epsilon) \mathcal{O} (t_1) - Q(t-\epsilon) \mathcal{O} (t_1) \rangle = \langle \delta \mathcal{Q} (t_1)    \]
But recall that the expectation value must be time-ordered! So:
\[      i\langle 0 | [ Q ,  \mathcal{O} ] | 0 \rangle = \langle 0 | \delta \mathcal{O} | 0 \rangle      \]
We're not doing this now, but you can show this holds for any state, not only the vacuum. So:
\[        i [Q, \mathcal{O} ] = \delta \mathcal {O}       \]
For example, for time translation the conserved charge is the Hamiltonian, and its commutator with an operator is the variation of that operator.



\end{document}































