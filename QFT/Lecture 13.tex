\documentclass[12 pt]{article}
\usepackage{amsmath,amssymb,amsthm,fullpage,amsfonts,enumerate,textcomp, eurosym}
\title{QFT Lecture 13}
\author{Matei Ionita}

\begin{document}
  \maketitle

Recall loop stuff.
\[   \frac{1}{i} \tilde \Delta_{full} (k^2) = \frac{1}{i} \frac{1}{k^2+m^2 - \pi(k^2) -i\epsilon}  \]
\[  \frac{ (ig)^2}{2i^2} \int \frac{d^d l}{(2\pi)^d} \frac{1}{l^2+m^2-i\epsilon}  \frac{1}{(l+k)^2+m^2-i\epsilon} - i (Ak^2 + Bm^2) \]
Impose:
\\
i) $\pi(-m^2) = 0$ such that m is the physical mass
\\
ii) $\pi'(-m^2) = 0$ such that the propagator has the correct normalization as $k^2 \to -m^2$.
\\
\\
We did the loop integral above by applying the Feynman trick, and then Wick rotation (to Euclidean time). The answer that we got was:
\[  \pi(k^2) = \left[ \frac{g^2}{2} \int_0^1 dx \int \frac{d^d \bar q}{(2\pi)^d} \frac{1}{(\bar q^2 + D)^2} \right]  - (Ak^2 + Bm^2)   \]
\[   \int \frac{d^d \bar q}{(2\pi)^d} \frac{(\bar q^2)^a}{(\bar q^2 + D)^b}  = \frac{\Gamma(b-a-d/2) \Gamma(a+d/2)}{(2\pi)^{d/2} \Gamma(b) \Gamma(d/2)} D^{-(b-a-d/2)} \]
The gamma function at special points:
\[  \Gamma(n) = (n-1)!   \]
\[  \Gamma(n+1/2) = \frac{(2n)!}{n! 4^n} \sqrt(\pi)  \]
Poles at $x=0$ or $x=-n$; asymptotic dependence:
\[  \Gamma(x)_{x\to 0} \sim \frac{1}{x} - \gamma + O(x)    \]
\[   \Gamma(-n+x)_{x\to 0} \sim \frac{(-1)^n}{n!}\left( \frac{1}{x} -\gamma + 1 + \frac{1}{2} + ... + \frac{1}{n} + O(x)   \right)   \]
Using these formulas, the result of our loop integral is:
\[  \frac{\Gamma(2-d/2)}{(4\pi)^{d/2}} D^{-(2-d/2)}   \]
We are interested in $d=6$, where this diverges. We will pretend therefore that $d=6-\epsilon$:
\[    \frac{\Gamma(-1+\epsilon/2)}{(4\pi)^{3-\epsilon/2}} D^{1-\epsilon/2}   \]
We will therefore say that:
\[  \int \frac{d^d \bar q}{(2\pi)^d} \frac{1}{(\bar q^2 + D)^2} = \frac{1}{(4\pi)^3} \left( 1+\frac{\epsilon}{2} \rm{ln} (4\pi) + O(\epsilon^2)\right) (-1) \left(   \frac{2}{\epsilon} - \gamma + 1 + O(\epsilon) \right) \]
\[ D \left( 1-\frac{\epsilon}{2} \rm{ln}D + O(\epsilon^2)  \right)  = \]
\[     = \frac{D}{(4\pi)^3} \left(  \frac{2}{\epsilon} - \gamma + 1 + \rm{ln} 4\pi - \rm{ln}D + O(\epsilon)  \right)    \]
This is called method of dimensional regularization. We can rewrite the answer as:
\[   \frac{D}{(4\pi)^3} \left(  \frac{2}{\epsilon}  + 1 + \rm{ln} (4\pi/e^{\gamma}) - \rm{ln}D + O(\epsilon)  \right)      \]
\[     \pi(k^2) =- \frac{g^2}{2 (2\pi)^3}  \left[    \int_0^1 dx D (2/\epsilon + 1 +\rm{ln}(4\pi/e^{\gamma})  - \int_0^1 dx D \rm{ln}D  \right]  - (Ak^2 + Bm^2) \]
The first integral gives:
\[  \left( \frac{k^2}{6} + m^2 \right) \left( \frac{2}{\epsilon} + 1 + \rm{ln} \frac{4\pi}{e^{\gamma}} \right) \]
Which explains the fact that we want to substract terms proportional to $k^2$ and $m^2$. This tells us that $A$ and $B$ should contain terms:
\[   A = Z_{\phi} - 1 = - \frac{1}{6} \frac{g^2}{(4\pi)^3} \frac{1}{\epsilon} + ...    \]
\[   B = Z_m - 1  = - \frac{g^2}{(4\pi)^3} \frac{1}{\epsilon} + ...        \]
Where the $...$ represent non-divergent terms. Call these a and b, and the leftovers are:
\[   \pi(k^2) = \frac{q^2}{2 (2\pi)^3} \int_0^1 dx D \rm{ln}D  + (ak^2 + bm^2)    \]
In order to satisfy $\pi(-m^2) = 0 , \pi'(-m^2) = 0$ we get:
\[   \pi(k^2)   = \frac{g^2}{2(4\pi)^3} \int_0^1 dx D \rm{ln} \frac{D}{D_0}  -  \frac{1}{12} \frac{g^2}{(4\pi)^3} (k^2 + m^2)  \]
Where $D_0 = D(k^2 = -m^2)$.
\\
\\
Read chapter 14. On Thursday we do chapter 16.

\end{document}