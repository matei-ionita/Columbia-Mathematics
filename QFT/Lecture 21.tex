\documentclass[12 pt]{article}
\usepackage{amsmath,amssymb,amsthm,fullpage,amsfonts,enumerate,textcomp, eurosym}
\title{QFT Lecture 21}
\author{Matei Ionita}
\DeclareMathOperator {\p} {\partial}

\begin{document}
  \maketitle

\subsection*{Spinor fields}
References: Srednicki 33-35, Zee II.3, appendix E. We'll only talk about spinors in 4 dimensions. Also look at Polchinski II. Appendix about how to construct spinors in arbitrary dimensions.
\\
\\
Let's remember what scalar means. $x\to \Lambda x$ leads to $\phi(x) \to \phi(\Lambda^{-1} x)$. (I.e. $\phi$ at some point P has the same value in the Lorentz-transformed coordiantes). For a vector field, we would have $A^{\mu} (x) \to \Lambda^{\mu}_{\nu} A^{\nu} (\Lambda^{-1} x)$. We can wonder if there exists some object in between scalar and vector. Look at an object $\psi_a$ with arbitrary number of components. We make it transform like: $\psi_a (x) \to L_a^b (\Lambda) \psi_b (\Lambda^{-1} x) $. In operator language: $U(\Lambda)^{-1} \hat \psi_a U(\Lambda) = L_a^b (\Lambda) \hat \psi_b (\Lambda^{-1} x)$. Some properties we would like $L_a^b$ to have are:
\\
\\
1. \[ L_a^b (\Lambda = 1) = \delta_a^b \]
2. \[ U(\Lambda_2)^{-1} \left[ U(\Lambda_1)^{-1} \hat \psi_a U(\Lambda_1) \right] U(\Lambda_2) = L_a^b (\Lambda_1) L_b^c (\Lambda_2) \hat \psi_c (\Lambda^{-1} x) \]
\[ U(\Lambda_1\Lambda_2)^{-1}  \hat \psi_a  U(\Lambda_1\Lambda_2) = L_a^b (\Lambda_1) L_b^c (\Lambda_2) \hat \psi_c (\Lambda^{-1} x) \]
Thus: $L_a^c (\Lambda_1 \Lambda_2) = L_a^b (\Lambda_1) L_b^c (\Lambda_2)$, so the set of $L$ matrices has the same group structure as Lorentz transformations. Therefore the $L$ matrices form a representation of the Lorentz group. By finding all representations of $SO(1,3)$, we get all particles.
\\
\\
Example: trivial representation: $\forall \Lambda, L_a^b(\Lambda) = \delta_a^b$. Vector representation: $L_a^b(\Lambda) = \Lambda_a^b$.
\\
\\
Expand $L_a^b$:
\[     L_a^b (\Lambda) = \delta_a^b + \frac{i}{2} \delta \omega_{\mu\nu} (S^{\mu\nu})_a^b    \]
We will recover $K^i = S^{i0}$ and $J^i = \frac{1}{2} \epsilon_{ijk} S^{jk}$. Earlier in the semester we computed the generators for the Lorentz group: $[J^i, J^j] = i \epsilon_{ijk} J^k$, $[J^i,K^j] = i\epsilon_{ijk} K^k$, $[K^i, K^j] = i\epsilon_{ijk} J^k$.
\\
\\
Representations of $J$ are labeled by spin. In the spin $1/2$ representation, they are the Pauli matrices. How do we find all representations of the Lorentz group? Wigner defined $N_i = \frac{1}{2} (J_i - iK_i)$ and $N_i^{\dagger} = \frac{1}{2} (J_i + iK_i)$. With these definitions, we get the Lie algebra: $[N_i, N_j] = i \epsilon_{ijk} N_k$, $[N^{\dagger}_i, N^{\dagger}_j] = i \epsilon_{ijk} N^{\dagger}_k$, $[N_i , N_j^{\dagger}] = 0$. We thus have two copies of $SU(2)$; label them by $j_L, j_R$. For $j_L = j_R = 0$ we get the 1-1 repsesentation, i.e. a scalar.
\\
\\
Let's look at $j_L = 1/2, j_R = 0$, called a 2-1 representation. If we rotate along $N$, the object rotates, but about $N^{\dagger}$ it behaves like a scalar. In other words, $N^{\dagger} = 0$ for this representation. We need one index, so: $\psi_a \to L_a^b \psi_b$. $N^{\dagger} = 0 \Rightarrow K_i = iJ_i$. Therefore $J_i =  \sigma_i/2$ and $K_i = i \sigma_i /2$.



\end{document}