\documentclass[12 pt]{article}
\usepackage{amsmath,amssymb,amsthm,fullpage,amsfonts,enumerate,textcomp, eurosym}
\title{QFT Lecture 9}
\author{Matei Ionita}

\begin{document}
  \maketitle

\subsection*{Perturbation theory}
\[  S = \int d^4 x (\mathcal{L}_0 + \mathcal{L}_{int} )  \]
\[  \mathcal{L}_int = -\frac{1}{2} (Z_{\phi} -1)(\partial \phi)^2 - \frac{1}{2} (Z_m -1) m^2 \phi^2 + \frac{1}{3!}Z_g g \phi^3 +Y\phi + \Lambda  \]
\[  Z_1[J] = \int D\phi e^{i\int d^4x \mathcal{L}_0 + 1/3! Z_g g \phi^3 + J\phi}  \]
\[  Z_1[J] = \sum_{V=0}^{\infty} \frac{1}{v!} \left(\frac{i}{3!} Z_g g \int d^3 x (\delta/i \delta J_x)\right)^v  \sum_{P=0}^{\infty} \frac{1}{P!} \left( \frac{1}{2} \int d^4 y d^4 z iJ_y iJ_z \Delta_{yz} /i \right) ^p \]
\[ Z_1[J] = e^{iW[J]}  \]
Where $W[J]$ is the sum over all connected diagrams. For example:
\[   \Theta = \frac{1}{12} (iZ^g g)^2 \int d^4x d^4 y \left( \frac{\Delta_{xy}}{i} \right)^3  \]
\[   dumbell = \frac{1}{8} (iZ^g g)^2 \int d^4x d^4 y  \frac{\Delta_{xy}}{i} \left(  \frac{\Delta(0)}{i}  \right) ^2 \]
\[  O-*  = \frac{1}{2} (iZ_g g) \int d^4x d^4 y (iJ_y) \frac{\Delta_{xy}}{i} \frac{\Delta(0)}{i}    \]
\[  *-* = \frac{1}{2} \int d^4x d^4 y (iJ_x)(iJ_y) \frac{\Delta_{xy}}{i}     \]
\[  *-O-* = \frac{1}{4} (iZ_g g)^2 \int d^4x d^4 y d^4 a d^4 b (iJ_a)(iJ_b) \left(  \frac{\Delta_{xy}}{i}   \right)^2 \frac{\Delta_{ax}}{i} \frac{\Delta_{yb}}{i}    \]
Let's focus for a bit on diagrams with no J. For the free generating function:
\[   Z_0[J=0] = 1  \]
With $\phi^3$ interaction, this no longer holds:
\[   Z_1[J=0] = e^{iW[J=0]}   \]
i.e. the exponential of the sum of all bubble diagrams. This is not 1 in general. To correct this, we introduce the counterterm $\Lambda$. To get 1, we have to choose:
\[   -i \int d^4x \Lambda = \Theta + dumbell + ...  \]
\[   \int d^4 x \Lambda = - W(J=0)  \]
It's not obvious why this works - why the sum over all bubbles is the volume integral of a constant. To see it, let's look at the $\Theta$ diagram:
\[   \Theta = \frac{1}{12} (iZ^g g)^2 \int \frac{d^4x d^4 y}{i^3} \left( \int \frac{d^4 k_1}{2\pi^4} \frac{e^{ik_1 (x-y)}}{k_1^2 +m^2 -i\epsilon}  \right) (2)(3)   \]
\[   \Theta = \frac{1}{12} (iZ^g g)^2 \int \frac{d^4x}{i^3} \int \frac{d^4 k_2 d^4 k_3}{2\pi^8} \frac{1}{(-k_2-k_3)^2 +m^2 -i\epsilon}\frac{1}{k_1^2 +m^2 -i\epsilon}\frac{1}{k_1^2 +m^2 -i\epsilon}     \]
The k integral goes like $k^2$ - diverges. Thus $\Lambda$ is infinite. To go around this issue, we introduce the ULTRAVIOLET CUTOFF: integrate only up to $k = k_{UV}$. The infinite $\Lambda$ problem is very bad because it appears with a - sign in $\mathcal{H}$. We get a $-\lambda$ energy density for vacuum. Intuitively, the bubble diagrams represent vacuum energy because only J terms are interactions.
\\
\\
Now we'll make the argument that we also want to cancel out $1 J$ diagrams. Consider:
\[  \langle 0| \hat \phi(x) |0\rangle = \left. \frac{\delta Z_1[J]}{i\delta J(x)}\right|_{J=0} = e^{iW[J]} \left. \frac{\delta iW(J)}{\delta i J_x} \right|_{J=0}  \]
The first factor is already 1 by the removal of bubbles. The second factor Gives the derivative of all diagrams with one J. Why do we want this to be 0? This is:
\[   \langle 0| \hat \phi(x) |0\rangle = \langle 0| e^{iHt} \hat \phi(0, \mathbf{x}) e^{-iHt} |0\rangle  \]
We can make the exponentials act on the vacuum states, which they leave unchanged. Therefore the 1-point function should be time independent. Then we can take the time to be in the far future or far past, where the field behaves like a free field. Using the expansion of the free field in terms of $a$ and $a^{\dagger}$, $a$ annihilates the ket and $a^{\dagger}$ annihilates the bra. We get 0. An intuitive argument is that one J diagrams mean that a particle interacts with the vacuum; if we want a stable vacuum  (one that doesn't randomly poop out particles), we make them 0. For this reason, we introduce the $Y\phi$ term. The vertices will have 1 leg sticking out, so there's only one possible diagram:
\[   \rm{x}-* = (iY) \int d^4x d^4y (iJ_y) \Delta_{xy}/i   \]
We want this to cancel out the diagram:
\[  O-*  = \frac{1}{2} (iZ_g g) \int d^4x d^4 y (iJ_y) \frac{\Delta_{xy}}{i} \frac{\Delta(0)}{i}    \]
We make the choice:
\[   Y = \frac{-Z_g g}{2} \frac{\Delta(0)}{i}  \]
This is the $O(g)$ contribution to Y. Choose $O(g^3)$ contributions to Y to cancel out all the $g^3$ one J diagrams. Then we can do this order by order. Shaded sphere - placeholder for the sum over EVERYTHING. 



\end{document}






































