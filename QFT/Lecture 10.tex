\documentclass[12 pt]{article}
\usepackage{amsmath,amssymb,amsthm,fullpage,amsfonts,enumerate,textcomp, eurosym}
\title{QFT Lecture 10}
\author{Matei Ionita}

\begin{document}
  \maketitle

Now that we know how to compute $Z[J]$, let's see how we get scattering amplitudes. We take derivatives of the generating function and use the LSZ formula. We have:
\[   Z[J] = e^{iW[J]}  \]
Throwing away bubbles and tadpoles in $W[J]$. Let's start with the 2-point function, the simplest n-point function that is not 0.
\[   \langle \phi_1 \phi_2 \rangle = \frac{1}{i} \Delta_{exact}  \]
We denote the interaction propagator by $\Delta_{exact}$, and the free propagator by $\Delta$.
\[   \langle \phi_1 \phi_2 \rangle =  \left. \frac{\delta Z[J]}{\delta i J_2 \delta i J_1} \right|_{J=0} = \frac{\delta}{\delta iJ_2}\left. e^{iW[J]}\frac{\delta i W[J]}{ \delta i J_1} \right|_{J=0} =\left. \frac{\delta iW}{\delta iJ_1 \delta i J_2} \right|_{J=0}  \]
\[  =  \frac{\delta }{\delta iJ_1 \delta i J_2} \left[  *-* + *-O-* + ...   \right] \]
Where the first diagram is $O(g^0)$, the second is $O(g^2)$ etc.
\[  *-* = \frac{1}{2} \int d^4 x d^4 y (iJ_x) (iJ_y) \frac{\Delta_{xy}}{i}   \]
\[   \frac{\delta }{\delta iJ_1 \delta i J_2}  *-* = \frac{1}{i} \Delta_{xy}  \]
To the lowest order in g, we get the free 2-point function, which makes sense. Let's do the next term.
\[   *-O-* = \frac{1}{4} \int d^4 x d^4 y d^4 a d^4 b (iZ_g g)^2 \left(  \frac{\Delta_{xy}}{i}   \right)^2 \frac{\Delta_{xa}}{i} \frac{\Delta_{yb}}{i} (iJ_a) (iJ_b) \]
We'll see next week that $Z_g = 1 + O(g^2)$, so for the purposes of the $O(g^2)$ term it's 1.
\[    \frac{\delta }{\delta iJ_1 \delta i J_2} *-O-* = \frac{(ig)^2}{2} \int d^4 x d^4 y \left( \frac{\Delta_{xy}}{i} \right)^2 \frac{\Delta_{1x}}{i} \frac{\Delta_{2y}}{i} = \frac{\Delta_{12}^{full}}{i} \]
Note that "full" may not be the best word here, since it's just the $O(g^2)$ term. Also note that, whenever we have a loop in the diagram, the integral diverges. We will see next week why this happens and how we can deal with it.
\\
\\
Now let's do the 4-point function, which we will use to compute 2-2 scattering.
\[  \frac{\delta }{\delta iJ_1 \delta i J_2 \delta iJ_3 \delta i J_4} e^{iW[J]} =   \frac{\delta }{\delta iJ_4 \delta i J_3} \left(  e^{iW} \frac{\delta i W[J]}{ \delta i J_1} \frac{\delta i W[J]}{ \delta i J_1} + e^{iW} \frac{\delta i W[J]}{ \delta i J_1 \delta i J_2}  \right)  =  \]
\[   = \frac{\delta i W[J]}{ \delta i J_1 \delta i J_2 \delta i J_3 \delta i J_4}  +  \frac{\delta i W[J]}{ \delta i J_1 \delta i J_2} \frac{\delta i W[J]}{ \delta i J_3 \delta i J_4} + \frac{\delta i W[J]}{ \delta i J_1 \delta i J_3} \frac{\delta i W[J]}{ \delta i J_2 \delta i J_4} + \frac{\delta i W[J]}{ \delta i J_1 \delta i J_4} \frac{\delta i W[J]}{ \delta i J_2 \delta i J_3}   \]
Note that the last 3 terms are products of 2-point functions. The first term we usually denote as:
\[  \langle \phi_{1'} \phi_{2'} \phi_1 \phi_2 \rangle _C   \]
Where the C stands for "connected"; a different type of connectedness than that of connected diagrams. The term that we have corresponds (to lowest order) to:
\[   :>-<: \]
It is a tree diagram (no loops); we'll focus on it for the rest of this lecture.
\[  :>-<: = \frac{(ig)^2}{2} \int d^4 x d^4 y d^4 a d^4 b d^4 c d^4 e (iJ_a) (iJ_b) (iJ_c) (iJ_e) \frac{\Delta_{xy}}{i} \frac{\Delta_{xa}}{i} \frac{\Delta_{xb}}{i} \frac{\Delta_{cy}}{i} \frac{\Delta_{dy}}{i} \]
We have three topologically distinct pairings:
\\
\\
\\
\\
\\
\\
\\
Now let's put these into the LSZ formula and compute the scattering amplitude. Recall the formula:
\[   _{OUT}\langle k_{1'} k_{2'} | k_1 k_2 \rangle_{IN} -  \{\rm{trivial\;scattering}\} =  i^4 \int d^4x_1 d^4 x_2 d^4 x_{1'} d^4 x_{2'} e^{i(k_1x_1 + k_2x_2 - k_{1'}x_{1'} - k_{2'}x_{2'})}\]
\[   (-\partial_1^2 +m^2) (-\partial_2^2 +m^2) (-\partial_{1'}^2 +m^2) (-\partial_{2'}^2 +m^2) \langle \phi_1 \phi_2 \phi_{1'} \phi_{2'} \rangle \]
Remark: we neglected the 2-point function terms above, but it turns out they give 0 in the LSZ formula anyway. This makes sense, since they only have 2 coupling vertices, so they couldn't represent an interesting interaction between two particles. The reason they vanish in LSZ is that the $k^2+m^2$'s are just zeros for real particles. But the Fourier transform of the interesting n-point functions has poles which cancel out the zeros. This doesn't happen for the 2-point functions. Let's check for free theory:
\[   \int d^4x_1 d^4 x_2 d^4 x_{1'} d^4 x_{2'} e^{i(k_1x_1 + k_2x_2 - k_{1'}x_{1'} - k_{2'}x_{2'})} \left[  \frac{\Delta{12}}{i}\frac{\Delta{1'2'}}{i} + \frac{\Delta{11'}}{i}\frac{\Delta{22'}}{i} + \frac{\Delta{12'}}{i}\frac{\Delta{21'}}{i} \right]   \]
Each propagator has a $q^2+m^2$ in the denominator, so we have 2 poles and 4 zeros, i.e. 0 overall. No nontrivial scattering for free theory - this is good. The claim is that 2-point functions give 0 even for interacting theories, but we will prove that later. The Fourier transform of the connected 4-point function is:
\[    \int d^4x_1 d^4 x_2 d^4 x_{1'} d^4 x_{2'} e^{i(k_1x_1 + k_2x_2 - k_{1'}x_{1'} - k_{2'}x_{2'})} \frac{(ig)^2}{i^5} \int d^4x d^4 y  \]
\[   \int \frac{d^4 p_a}{(2\pi)^4} \frac{e^{ip_a(x-y)}}{p_a^2+m^2} \int \frac{d^4 p_b}{(2\pi)^4} \frac{e^{ip_b(x-1)}}{p_b^2+m^2} \int \frac{d^4 p_c}{(2\pi)^4} \frac{e^{ip_c(x-2)}}{p_c^2+m^2} \int \frac{d^4 p_d}{(2\pi)^4} \frac{e^{ip_d(1'-y)}}{p_a^2+m^2} \int \frac{d^4 p_e}{(2\pi)^4} \frac{e^{ip_e(2'-y)}}{p_e^2+m^2} =  \]
\[ =    \frac{(ig)^2}{i^5} \int d^4x d^4 y  \int \frac{d^4 p_a}{(2\pi)^4} \frac{e^{ip_a(x-y)}}{p_a^2+m^2} \int \frac{d^4 k_1}{(2\pi)^4} \frac{e^{ik_1(x-1)}}{k_1^2+m^2} \int \frac{d^4 k_2}{(2\pi)^4} \frac{e^{ik_2(x-2)}}{k_2^2+m^2} \int \frac{d^4 k_{1'}}{(2\pi)^4} \frac{e^{ik_{1'}(1'-y)}}{k_{1'}^2+m^2} \]
\[ \int \frac{d^4 k_{2'}}{(2\pi)^4} \frac{e^{ik_{2'}(2'-y)}}{k_{2'}^2+m^2}  =  \]
\[  = (2\pi)^4 \delta(k_1+k_2 - k'_1 - k'_2) (ig)^2 \frac{1}{i} \frac{1}{(k_1+k_2)^2 + m^2 - i \epsilon}   \]
Notice that the leftover propagator can be thought of as the mediating particle between the vertices. The other 4 propagators were canceled out by the LSZ formula. Note that the mediating particle is a "virtual particle", i.e. it is not on-shell. Feynman rules:
\\
i) always get a delta function enforcing momentum conservation
\\
ii) always have a propagator for the internal particle
\\
iii) always get the coupling constant at the appropriate (i.e. no of particles) power
\\
By thinking about the mediating particle, we can write the propagators for the other 2 tree diagrams in the process:
\[  (2\pi)^4 \delta(k_1+k_2 - k'_1 - k'_2) (ig)^2 \left[ \frac{1}{i} \frac{1}{(k_1+k_2)^2 + m^2 - i \epsilon}  + \frac{1}{i} \frac{1}{(k_1-k_1')^2 + m^2 - i \epsilon} +\right. \]
\[ \left. + \frac{1}{i} \frac{1}{(k_1-k_2')^2 + m^2 - i \epsilon} \right] \]
We can redefine the scattering amplitude as the above with the delta function removed:
\[    _{OUT}\langle k_{1'} k_{2'} | k_1 k_2 \rangle_{IN} -  "1" = (2\pi)^4 \delta(k_1+k_2 - k'_1 - k'_2) i\mu    \]
Now we compute the cross-section from the scattering amplitude (see Srednicki 11).







\end{document}






















