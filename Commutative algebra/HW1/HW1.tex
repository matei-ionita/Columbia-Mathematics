\documentclass[12 pt]{article}
\usepackage{amsmath,amssymb,amsthm,fullpage,amsfonts,enumerate,textcomp, eurosym, tikz-cd, fullpage}
\title{Commutative algebra HW1}
\author{Matei Ionita}

\DeclareMathOperator {\p} {\partial}
\DeclareMathOperator {\R} {\mathbb{R}}
\DeclareMathOperator {\C} {\mathbb{C}}
\DeclareMathOperator {\Q} {\mathbb{Q}}
\DeclareMathOperator {\Z} {\mathbb{Z}}


\begin{document}
  \maketitle

\subsection*{Problem 1}
Let $A = k[x, y]/(x^2y^4 - x^4y^2 + 1)$ where $k$ is a field. Construct a $k$-algebra map as in Noether Normalization for $A$. Very briefly explain why it works.
\\
\\
\emph{Solution}
\\
As in the proof of Noether Normalization, we want to make a change of variables $(x,y) \to (z,y)$ such that $y$ is integral over $k[z]$. We let $z = x - y^2$, and it turns out that the exponent 2 is large enough in order to make the resulting polynomial monic:
\[     x^2y^4 - x^4y^2 + 1  \to  1+y^8-y^{10}+2 y^6 z-4 y^8 z+y^4 z^2-6 y^6 z^2-4 y^4 z^3-y^2 z^4  \]
Then by Lemma 1 proved in class $\phi :k[z] \to k[x,y]/(x^2y^4 - x^4y^2 + 1)$ defined by $\phi(z) = x-y^2$ is a finite map.

\subsection*{Problem 2}
Describe all prime ideals of $\C[x, y]/(xy)$ where $\C$ is the field of complex numbers. Just list them in some way and explain briefly why they are primes and why you've got all of them.
\\
\\
\emph{Solution}
\\
As discussed in class, there is a 1-1 correspondence between ideals of $\C[x,y] / (xy)$ and ideals of $\C[x,y]$ that contain $(xy)$. Moreover, if one of the latter is prime, then so is its correspondent. We will therefore consider prime ideals in $\C[x][y] = \C[x,y]$. Since $\C[x]$ is a PID, its prime ideals are of 3 types:
\\
\\
1) the 0 ideal, which does not contain $(xy)$.
\\
2) principal ideals $\big(f(y)\big)$ generated by a polynomial $f$ that is irreducible in $\C[x]$. The only such ideal that generates $(xy)$ is $(y)$.
\\
3) $\big( p, f(y)  \big)$ where $p$ is prime in $\C[x]$ and $f$ is irreducible in $\C[x]/(p)$. The only such ideals that generate $(xy)$ are $(x-\lambda , y), (x, y-\mu)$ and $ (x)$, for all $\lambda, \mu \in \C$.
\\
\\
It follows that all prime ideals of $\C[x,y] / (xy) $ are $\big( x + (xy) \big)$, $\big( y + (xy) \big)$, $\big( x - \lambda + (xy) , y+ (xy) \big)$, $\big( x + (xy) , y- \mu \big)$.



\subsection*{Problem 3}
Let $k$ be a field. Prove that $k[x, y]$ is not isomorphic to $k[x, y, z]$.
\\
\\
\emph{Solution}
\\
Assume the two rings are isomorphic; then obviously the isomorphism is a finite map. Then there exist $n$ generators that generate $k[x,y,z]$ as a $k[x,y]$ - module. This construction must span $z, z^2, ... , z^{n+1}$, all of which are linearly independent over $k[x,y]$. But this means that at least $n+1$ generators are required, which is a contradiction.


\subsection*{Problem 4}
Let $k$ be a field. Suppose $A$ is a $k$-algebra and $f$ is a nonzerodivisor of $A$ such that $k[x, y]$ is isomorphic to $A_f$ as $k$-algebra. Show that $A$ is isomorphic to $k[x, y]$.
\\
\\
\emph{Solution}
\\
Since isomorphisms preserve units, $f/1$ and $1/f$ are units in $k[x,y]$, therefore $f/1 , 1/f \in k$. Since every map from a field to a ring is injective, the ring map $k\to A$ is an inclusion map, and therefore $f, f^{-1} \in A$. This means that localization does nothing to $A$, since we are inverting something which is already invertible. Thus $A \cong A_f \cong k[x,y]$.

\subsection*{Problem 5}
Let $A$ be a ring and let $f$ be an element of $A$. Show that $A_f$ is as an $A$-algebra isomorphic to $A[x]/(fx - 1)$.
\\
\\
\emph{Solution}
Consider the following maps:
\[
\begin{tikzcd}
A\arrow[swap]{d}{\phi}\arrow{dr}{g}  \\
A_f & A[x]/(fx-1)
\end{tikzcd}
\]
$\phi(a) = a/1$ and $g(a) = a$. The by the universality of localizations (see for example corollary 3.2 in Atiyah - Macdonald) there exists and isomorphism $h:A_f \to A[x]/(fx-1)$. The diagram above is easily shown to satisfy the requirements for the universality theorem:
\\
(a) $s\in \{f, f^2, ... \} \Rightarrow g(s)$ is a unit in $A[x]/(fx-1)$
\\
(b) $g(a) = 0 \Rightarrow as = 0$ for some $s\in \{f, f^2, ... \}$
\\
(c) every element of $A[x]/(fx-1)$ is of the form $g(a)\big(g(s)\big)^{-1}$
\\
The first two are obvious, the third follows since $X = g(f)^{-1}$.


\end{document}























