\documentclass[12 pt]{article}
\usepackage{amsmath,amssymb,amsthm,fullpage,amsfonts,enumerate,textcomp, eurosym, tikz-cd, fullpage}
\title{Commutative algebra HW3}
\author{Matei Ionita}

\DeclareMathOperator {\p} {\partial}
\DeclareMathOperator {\R} {\mathbb{R}}
\DeclareMathOperator {\C} {\mathbb{C}}
\DeclareMathOperator {\Q} {\mathbb{Q}}
\DeclareMathOperator {\Z} {\mathbb{Z}}
\DeclareMathOperator {\fr} {\mathfrak}


\begin{document}
  \maketitle

\subsection*{Problem 1}
\emph{Find a ring $A$ and an ideal $I$ such that $I$ is generated by countably many elements $f_1, f_2, f_3, ...$ such that $f_i^2 = 0$ but such that $I$ is not a nilpotent ideal (in other words for all $n > 0$ the ideal $I^n$ is not zero).}
\\
\\
\emph{Solution}
\\
Take $A = \C[x_1, x_2, \dots]/(x_1^2, x_2^2, \dots)$, and $I = (x_1, x_2, \dots)$. Then obviously each of the $x_i$'s is nilpotent, but $I^n$ will contain the element $x_1 \dots x_n \neq 0$.

\subsection*{Problem 2}
\emph{Let $A \subset B$ be an extension of domains. Let $K$ be the fraction field of $A$ and $L$ be the fraction field of $B$, so that we have an extension of fields $K \subset L$. Show that if (a) $B$ is a finite type $A$-algebra and (b) $L$ is a finite extension of $K$, then the image of $Spec(B) \to Spec(A)$ contains a nonempty open subset of $Spec(A)$.}
\\
\\
\emph{Solution}
\[
\begin{tikzcd}
B\arrow[hook]{r} & L \\
A\arrow[hook]{u}{\phi}\arrow[hook]{r} & K\arrow[hook]{u}
\end{tikzcd}
\]
Let $b \in B$, then $b\in L$, and since $L$ is a finite extension of $K$ we have $k_n b^n + \dots + k_0 = 0 $ for some $n$. By cancelling denominators, we can get this to the form $a_n b^n + \dots + a_0 = 0$ for $a_i \in A$, but we do not know that $a_n$ has an inverse, so this polynomial may not be monic. But we can localize $a_n$, in order to make $b$ integral. Therefore, if $(x_1, \dots, x_n)$ generate $B$ as an $A$-algebra, let $c_i$ denote the leading coefficient in the polynomial for $x_i$, and we localize $A$ and $B$ at the multiplicative subset $S = \{c_i\}$. Then $x_i \in S^{-1}B$ are integral over $S^{-1}A$, and this implies $S^{-1}\phi : S^{-1}A \to S^{-1}B$ is a finite map. The localization functor is exact, so $S^{-1}\phi$ is also injective. By Lemma 15 proved in class, Spec$(S^{-1}\phi)$ is surjective. Then:
\[    \text{Spec}(S^{-1}A) = \text{Im Spec}(S^{-1}\phi) \subset \text{Im } \text{Spec}(\phi) \;\;\;\;\;\;\;\; (*) \]
But Spec$(S^{-1}A)$ contains all primes in Spec$(A)$ that avoid $S = \{c_i\}$, so:
\[       \text{Spec}(S^{-1}A) = \text{D}(c_1 \dots c_n)   \;\;\;\;\;\;\;\; (**)   \]
By (*) and (**), $\text{D}(c_1 \dots c_n) \subset  \text{Im } \text{Spec}(\phi)$.

\subsection*{Problem 3}
\emph{Let $k$ be a field. Let $f, g \in k[t]$ be two polynomials in a variable $t$ with coefficients in $k$. Show that there exists a nonzero two variable polynomial $P \in k[x, y]$ such that $P(f, g) = 0$ in $k[t]$.}
\\
\\
\emph{Solution}
\\
Consider the map:
\begin{align*}
\phi : k[x,y] &\to k[t] \\
  P(x,y) &\to P\big(f(t), g(t)\big)
\end{align*}
This is a ring homomorphism. Assume there is no nonzero $P$ such that $P\big(f(t), g(t)\big) = 0$, then $\phi$ is injective. We want to show that $\phi$ is also finite. We have that $k[t]$ is a finitely generated $k[x,y]$-algebra, with generator $t$. By Lemma 1 proved in class, if $t$ satisfies a monic equation of the form
\[        t^n + \phi(a_1) t^{n-1} + \dots + \phi(a_n) = 0        \]
With $a_i \in k[x,y]$, then $\phi$ is finite. But we see that $\phi(x) = f(t) = \sum_{i=0}^n b_i t^i$, and therefore:
\[       t^n + \sum_{i=0}^{n-1} b_i/b_n t^i + \phi(x)/b_n = 0           \]
So $\phi$ is a finite injective map. By Lemma 15, Spec$(\phi)$ is surjective, and then by Lemma 29 dim$(k[x,y])$=dim$(k[t])$. But this is a contradiction, since dim$(k[x,y]) = 2$ and dim$(k[t]) = 1$.


\subsection*{Problem 4}
\emph{Give an example of an Artinian ring which is not an algebra of finite type over a field.}
\\
\\
\emph{Solution}
\\
$\Z_{4}$ is Artinian because it has a finite number of ideals. It is not a finite-type algebra over a field because there exists no ring homomorphism from a field other than $\Z_4$ to $\Z_4$.



\end{document}






















