\documentclass[12 pt]{article}
\usepackage{amsmath,amssymb,amsthm,fullpage,amsfonts,enumerate,textcomp, eurosym, tikz-cd, fullpage}
\title{Commutative algebra HW5}
\author{Matei Ionita}

\newcommand{\R}{\mathbb{R}}
\newcommand{\Q}{\mathbb{Q}}
\newcommand{\Z}{\mathbb{Z}}
\newcommand{\F}{\mathbb{F}}
\newcommand{\C}{\mathbb{C}}
\newcommand{\CP}{\mathbb{C}\mathbb{P}}
\newcommand{\RP}{\mathbb{R}\mathbb{P}}
\newcommand{\Proj}{\mathbb{P}}
\newcommand{\N}{\mathbb{N}}
\newcommand{\p}{\partial}
\newcommand{\fr}{\mathfrak}

\DeclareMathOperator{\Hom}{Hom}
\DeclareMathOperator{\length}{length}
\DeclareMathOperator{\res}{Res}
\DeclareMathOperator{\Int}{Int}
\DeclareMathOperator{\Ext}{Ext}
\DeclareMathOperator{\Aut}{Aut}
\DeclareMathOperator{\Gal}{Gal}
\DeclareMathOperator{\Sym}{Sym}
\DeclareMathOperator{\Lie}{Lie}
\DeclareMathOperator{\id}{Id}
\DeclareMathOperator{\tr}{tr}
\DeclareMathOperator{\irr}{irr}
\DeclareMathOperator{\supp}{supp}
\DeclareMathOperator{\trdeg}{trdeg}
\DeclareMathOperator{\Spec}{Spec}
\DeclareMathOperator{\Nm}{Nm}


\begin{document}
  \maketitle

\subsection*{Problem 3}
\emph{Let $k = \C$ be the field of complex numbers. Compute the integral closure of the domain $k[x, y, z]/(z^6 - x^2 y^3)$ in its fraction field.}
\\
\\
\emph{Solution}
\\
Define $t = xy/z^2, s = z^3/xy$. Then we have $t^3 = x$ and $s^2 = y$, thus $s,t$ are integral over $A = k[x, y, z]/(z^6 - x^2 y^3)$. Then construct the map:
\begin{align*}       \phi : A &\to k[t,s]     \\
 k &\to k \\
x &\to t^3 \\
y &\to s^2 \\
z &\to ts
\end{align*}
And the rest follows from homomorphism properties. We need to show that this map is well-defined:
\[   \phi(z^6 - x^2 y^3) = (ts)^6 - t^6s^6 = 0    \]
We showed above that $s,t$ are integral over $A$, therefore $k[t,s]$ is integral over $A$. We claim that this is actually the integral closure of $A$. First, note that $k[t,s]$ is a UFD, so it's integrally closed over its fraction field. If we show that $k(t,s)$ is isomorphic to f.f.$(A)$, then $k[t,s]$ will be integrally closed over f.f.$(A)$, and so the claim is proved. So let $\tilde \phi$ be the obvious map between fraction fields that restricts to $\phi$ on $A$. Then $\phi$ is injective, since $\tilde \phi (a) = 1 = t^0s^0$ can only be achieved if the powers of $t$ and $s$ in the numerator and denominator cancel, which happens when $a =\left( \frac{z^6}{x^2y^3}\right)^n = 1^n = 1$ for some $n\in \Z$. To see that $\tilde \phi$ is also surjective, note that it suffices to show that the preimage of $t$ and $s$ is nonempty, since all integer powers of $s,t$ will be generated by homomorphism properties. And we have:
\[   \tilde \phi \left(  \frac{xy}{z^2} \right) = t \;\;\;\;\;\;\;\;\;  \tilde \phi \left(  \frac{z^3}{xy} \right) = s  \]
Thus the fraction fields are isomorphic, and this completes the proof.

\subsection*{Problem 4}
\emph{Give an example of a domain $R$ such that the integral closure of $R$ in its fraction field is not finite over $R$.}
\\
\\
\emph{Solution}
\\
We saw in class that $k[t]$ is the integral closure of $k[x,y]/(x^3 - y^2)$ over its fraction field. We can generalize this example as follows. Consider $\C[x_0, x_1, \dots]$, a polynomial ring in infinite variables. The ideals $(x_{2n}^3 - x_{2n+1}^2)$ are prime for all natural $n$. Moreover, any two of these ideals involves different variables, and so quotienting by one of them will not affect the primality of the others. Then the ring:
\[     R = \C[x_0, x_1, \dots] / (x_0^3 - x_1^2, x_2^3 - x_3^2 , \dots )    \]
is a domain. Just as in the example done in class, we can find fractions $t_n = \frac{x_{2n}}{x_{2n+1}}$ for all $n$ such that $t_n$ is integral over $R$. (In fact, $t_n^2 = x_{2n}$) The integral closure of $R$ will have to contain at least the $R$-independent fractions $t_n$, and since these are infinitely many, the integral closure is not finite over $R$.

\subsection*{Problem 6}
\emph{Let $k$ be a field and $B = k[x, y]$ with grading determined by $\deg(x) = 2$ and $\deg(y) = 3$. Compute the Hilbert function of $B$. Is there a Hilbert polynomial in this case?}
\\
\\
\emph{Solution}
\\
\[     B_n = \{ k_{ab} x^a y^b | 2a + 3b = n  \}     \]
Therefore $H_B(n)$ counts the number of natural solutions to the equation $2a+3b = n$. We compute this by hand for $n<6$:
\[    H_B(0) = 1 \;\;\;\; H_B(1) = 0 \;\;\;\;H_B(2) = 1 \;\;\;\;H_B(3) = 1 \;\;\;\;H_B(4) = 1 \;\;\;\;H_B(5) = 1     \]
We want to prove by induction that $H_B(6+n) = 1+ H_B(n)$, which would show that $H_B(6k+i) = k$ if $i=1$ and $k+1$ otherwise. I'm not sure how to prove this algebraically, so I'm attaching a graphical proof at the end of the problem set. We can regard solutions to the equation $2a + 3b = n$ as the points with natural coordinates on the line $b = n/3 - 2a/3$. The solid lines in the picture are the lines for $n \equiv 0$ mod $3$, and the dashed lines for all other $n$. Consider the red triangle; the lines it contains are the base case studied above, $n<6$. If we translate it 3 units to the right, to the other red triangle, the number and position of solutions it contains will not change. Then, for $6\leq n <12$, the only solutions we gain are in the parallelogram directly above the base triangle. But it's easy to see that each line has exactly one solution in this parallelogram. Therefore increasing $n$ by 6 gave one extra solution. The same argument can be repeated for the pink triangle, and in general any right triangle with legs that are multiples of 2 and 3; this completes the induction. Finally, let us note that there exists no Hilbert polynomial. The Hilbert function is always a linear function of $n$, but the $b$-intercept switches between 0 and 1 infinitely.

\subsection*{Problem 7}
\emph{Let $k$ be a field and $B = k[x, y]/(x^2, xy)$ with grading determined by $\deg(x) = 2$ and $\deg(y) = 3$. Compute the Hilbert function of $B$. Is there a Hilbert polynomial in this case?}
\\
\\
\emph{Solution}
\\
\[   B_0 = k \Rightarrow H_B(0) = 1   \]
\[   B_1 = \emptyset \Rightarrow H_B(1) = 0    \]
\[   B_2 = \{ ax | a\in k \}  \Rightarrow H_B(2) = 1 \]
For all $n>2$, the only polynomials allowed in $B_n$ will be powers of $y$, since powers of $xy$ and powers of $x$ greater than 1 are 0. Moreover, since $\deg(y) = 3$, we will have no degree $n$ elements for $n \equiv 1,2$ mod $3$, and $H_B(n) = 0$ in this case. For $n\equiv 0$ mod $3$, we have $B_n = \{ a y^{n/3} | a\in k \}$, so $H_B(n) = 1$. There is no Hilbert polynomial because the Hilbert function has infinitely many jumps between $0$ and $1$.

\subsection*{Problem 8}
\emph{Let $k$ be a field and $B = k[x, y, z]/(x^d + y^d + z^d)$ with grading determined by $\deg(x) = \deg(y) = \deg(z) = 1$. Compute the Hilbert function of $B$. Is there a Hilbert polynomial in this case?}
\\
\\
\emph{Solution}
\\
First we analyze the case $n<d$. None of the homogenous degree $n$ polynomials are quotiented out, and it is a standard result that in this case:
\[      H_B(n) = \binom{n+2}{2}     \]
Now we turn to $n\geq d$. We don't have $\binom{n+2}{2}$ linearly independent homogenous polynomials of degree $n$, because some of these are $0$ in the quotient. These are the polynomials that are multiples of $x^d + y^d + z^d$. All these look like $(x^d + y^d + z^d) f(x,y,z)$ for some $f$ which has degree $n-d$. Therefore counting them is equivalent to counting the polynomials of degree $n-d$, which are $\binom{n-d+2}{2}$. We obtain:
\[      H_B(n) = \binom{n+2}{2} - \binom{n-d+2}{2} = \frac{(n+2)(n+1)}{2} - \frac{(n-d+2)(n-d+1)}{2}   \]
\[   H_B(n) = dn + \frac{3d-d^2}{2}   \]
The Hilbert function is a linear polynomial for $n\geq d$.






\end{document}






































