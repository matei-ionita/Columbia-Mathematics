\documentclass[12 pt]{article}
\usepackage{amsmath,amssymb,amsthm,fullpage,amsfonts,enumerate,textcomp, eurosym, tikz-cd, fullpage}
\title{Commutative algebra HW12}
\author{Matei Ionita}

\newcommand{\R}{\mathbb{R}}
\newcommand{\Q}{\mathbb{Q}}
\newcommand{\Z}{\mathbb{Z}}
\newcommand{\F}{\mathbb{F}}
\newcommand{\C}{\mathbb{C}}
\newcommand{\CP}{\mathbb{C}\mathbb{P}}
\newcommand{\RP}{\mathbb{R}\mathbb{P}}
\newcommand{\Proj}{\mathbb{P}}
\newcommand{\N}{\mathbb{N}}
\newcommand{\p}{\partial}
\newcommand{\fr}{\mathfrak}

\DeclareMathOperator{\Hom}{Hom}
\DeclareMathOperator{\length}{length}
\DeclareMathOperator{\res}{Res}
\DeclareMathOperator{\Int}{Int}
\DeclareMathOperator{\Ext}{Ext}
\DeclareMathOperator{\Aut}{Aut}
\DeclareMathOperator{\Gal}{Gal}
\DeclareMathOperator{\Sym}{Sym}
\DeclareMathOperator{\Lie}{Lie}
\DeclareMathOperator{\id}{Id}
\DeclareMathOperator{\tr}{tr}
\DeclareMathOperator{\irr}{irr}
\DeclareMathOperator{\supp}{supp}
\DeclareMathOperator{\trdeg}{trdeg}
\DeclareMathOperator{\Spec}{Spec}
\DeclareMathOperator{\Nm}{Nm}
\DeclareMathOperator{\ord}{ord}


\begin{document}
  \maketitle

\section*{Problem 4}
We follow the proof of proposition 1.18 in ``The geometry of schemes'' by Eisenbud and Harris. First note that, by the result of Problem 7 below, $f_i$ generate the unit ideal iff the standard opens $D(f_i)$ cover $\Spec A$. Moreover, by the same problem we can assume that the $f_i$ are finitely many. 

Exactness at $A$ means that any global section that restricts to 0 on each $D(f_i)$ must be 0 on $A$. We prove this as follows. If the global section $g$ restricts to 0 on $D(f_i)$, then $g/1 = 0$ as an element of $A_{f_i}$. This means $g f_i^N = 0$ in $A$, for some power $N$. This holds for every $f_i$ (which are finitely many), so let $M$ be the maximal power among all these. Then the ideal generated by $f_i^M$ annihilates $g$. We use now the fact that $1 = \sum e_i f_i$ for some $e_i \in A$, whence $1 = 1^P = \left( \sum e_i f_i \right)^P$ for every power $P$. If we let $P$ be large enough, each term in the sum will contain some $f_i$ to a power greater or equal to $M$. Therefore the ideal generated by $f_i^M$ generates 1, and we see that $1 \cdot g = 0$. Therefore $g=0$.

Exactness at $\prod_i A_{f_i}$ means that the existence of $g_i$ on each $D(f_i)$ that agree on overlaps implies the existence of a global section $g$, such that $g|_{D(f_i)} = g_i$. To prove this, we want to construct a partition of unity that lets us glue together the local sections to form a global one. First note that $g_i \in A_{f_i}$ implies $f_i^N g_i \in A$ for large enough $N$. Since the $f_i$ are finitely many, let $M$ be the greatest such power, and define $h_i = f_i^M g_i \in A$. $g_i = g_j$ on $D(f_i, f_j)$, so on each overlap we have
\[          f_j^M h_i = f_j^M f_i^M g_i =  f_j^M f_i^M g_j = f_i^M h_j        \]
We will use this identity shortly. We will also use the fact that $\{ f_i \}$ generate 1, i.e.:
\[           1 = \sum_i e_i f_i      \]
Now define $g\in a$ as follows:
\[          g = \sum_i e_i h_i        \]
We show that indeed $g|_{D(f_j)} = g_j$. On each $D(f_j)$ we write:
\[         f_j^M g = \sum_i f_j^M  e_i h_i = \sum_i f_i^M e_i h_j = 1 \cdot h_j = f_j^M g_j         \]
But on $D(f_j)$ $f_j^M$ is a unit, so this implies $g = g_j$ as desired.

\section*{Problem 5}
Let $Z = \Spec \C[x] = \{ (0), (x-\lambda)\}$ and equip it with the structure sheaf. Consider the quotient $X = Z/ \sim$, where $\sim$ identifies $(x)$ and $(x-1)$. If $f: Z \to X$ is the natural projection, we can equip $X$ with the direct image sheaf $\mathcal{O}(U) = \mathcal{O}_Z (f^{-1}(U))$. The stalk $\mathcal{O}_x$ will be the union $\bigcup_{\lambda \neq 0,1} \C[x]_{x-\lambda}$. In this union all monomials except for powers of $x$ and $x-1$ have inverses. Therefore the stalk has two maximal ideals, $(x)$ and $(x-1)$.

\section*{Problem 6}
Take $X = \Spec \Z_{(p)}$, where $p$ is a prime, and equip it with the constant sheaf determined by $\Z_{(p)}$. All opens in $\Spec \Z_{(p)}$ are connected, so the ring above each of them is $\Z_{(p)}$. Then the stalk at each point is $\Z_{(p)}$. Do likewise for $Y = \Spec \C[[t]]$. Then $X$ and $Y$ are locally ringed spaces. Define a morphism as the map on spectra induced by the inclusion $\Z_{(p)} \to \C[[t]]$. The induced map on stalks is the inclusion $\Z_{(p)} \to \C[[t]]$. It is not a local map, because it takes the maximal ideal $(p)$ to constants in $\C[[t]]$. 


\section*{Problem 7}
We begin by showing that any collection of standard opens $\{D(f_i)\}$ covers $\Spec A$ iff $\{f_i\}$ generate 1. Note that $D(f_i)$ is the set of primes that don't contain $f_i$. These cover $\Spec A$ iff no prime $\fr p \in \Spec A$ contains all $f_i$. Now if $\{f_i\}$ generate 1, any ideal that contains all of them contains 1, therefore no prime contains all $f_i$. Conversely, assume that $\{f_i\}$ don't generate 1, then the ideal they generate is contained in some maximal $\fr m$. This $\fr m$ is then a prime that contains all $f_i$.

Now, to prove that $\Spec A$ is quasicompact, take an open cover $\bigcup U_i$ of $\Spec A$. Since the standard opens form a basis for $\Spec A$, this cover has a refinement $\bigcup_i D(f_i)$. As proved above, this is equivalent to $\{f_i\}$ generating 1. Then $1 = \sum_i e_i f_i$ for some coefficients $e_i \in A$. But this linear combination must be finite, and we obtain that finitely many of the $\{f_i\}$ generate 1. Using the first paragraph again, this is equivalent to $\bigcup_{\text{finitely many }i} D(f_i)$ being a finite subcover. Thus $\Spec A$ is quasicompact.

Every standard open $D(f_i)$ is the set of primes that don't contain $f_i$, and therefore $D(f_i) = \Spec A_{f_i}$. Then applying the above to this spectrum shows that $D(f_i)$ is quasicompact.




\end{document}






































