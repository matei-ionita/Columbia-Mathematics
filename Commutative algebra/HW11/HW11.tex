\documentclass[12 pt]{article}
\usepackage{amsmath,amssymb,amsthm,fullpage,amsfonts,enumerate,textcomp, eurosym, tikz-cd, fullpage}
\title{Commutative algebra HW11}
\author{Matei Ionita}

\newcommand{\R}{\mathbb{R}}
\newcommand{\Q}{\mathbb{Q}}
\newcommand{\Z}{\mathbb{Z}}
\newcommand{\F}{\mathbb{F}}
\newcommand{\C}{\mathbb{C}}
\newcommand{\CP}{\mathbb{C}\mathbb{P}}
\newcommand{\RP}{\mathbb{R}\mathbb{P}}
\newcommand{\Proj}{\mathbb{P}}
\newcommand{\N}{\mathbb{N}}
\newcommand{\p}{\partial}
\newcommand{\fr}{\mathfrak}

\DeclareMathOperator{\Mor}{Mor}
\DeclareMathOperator{\Hom}{Hom}
\DeclareMathOperator{\length}{length}
\DeclareMathOperator{\res}{Res}
\DeclareMathOperator{\Int}{Int}
\DeclareMathOperator{\Ext}{Ext}
\DeclareMathOperator{\Aut}{Aut}
\DeclareMathOperator{\Gal}{Gal}
\DeclareMathOperator{\Sym}{Sym}
\DeclareMathOperator{\Lie}{Lie}
\DeclareMathOperator{\id}{Id}
\DeclareMathOperator{\tr}{tr}
\DeclareMathOperator{\irr}{irr}
\DeclareMathOperator{\supp}{supp}
\DeclareMathOperator{\trdeg}{trdeg}
\DeclareMathOperator{\Spec}{Spec}
\DeclareMathOperator{\Nm}{Nm}
\DeclareMathOperator{\ord}{ord}

\begin{document}
  \maketitle

\subsection*{Problem 1}
Consider $F = X_0^2 X_1^2 + X_0^2 X_2^2 + X_1^2 X_2^2$. Dehomogenize on $\Proj^2 - V(X_0)$, to get:
\[         f(x_1, x_2) = x_1^2 + x_2^2 + x_1^2 x_2^2      \]
Cancelling the first derivatives gives:
\[     x_1 (1 + x_2^2) = 0      \]
\[     x_2 (1 + x_1^2) = 0              \]
Together with $f = 0$, we get the unique solution $x_1 = x_2 = 0$, which corresponds to $[1:0:0] \in \Proj^2$. Since $F$ is symmetric in $X_0, X_1, X_2$, we get a total of 3 singular points, $[1:0:0], [0:1:0]$ and $[0:0:1]$.

\subsection*{Problem 2}
The genus as a function of degree for a nonsingular curve is given by $g = \frac{1}{2}(d-1)(d-2)$, which in this case is 3. We also know that each singular point decreases the genus by at least 1, so $g\leq 0$. Therefore the only possibility is $g=0$.


\subsection*{Problem 3}
a) The product of the degrees, $2\cdot 3 = 6$, is an upper bound for the number of intersection points. To see that the bound is sharp, we show that the plane $X_3 = 0$ intersects $D$ in exactly 6 points. Setting $X_3 = 0$ we get:
\[         X_2^2 = - X_0^2 - X_1^2      \]
\[          X_2^3 = - X_0^3 - X_1^3    \]
And so:
\[       X_2 = \frac{X_0^3 + X_1^3}{X_0^2 + X_1^2}      \]
\[        ( X_0^2 + X_1^2)^3 + (X_0^3 + X_1^3)^2 = 0            \]
Using Mathematica, we see that the last equation has 6 different solutions in $\Proj^2$, as desired.
\\
\\
b) If $[X_0 : X_1 : X_2] \in D'$ has inverse image $[X_0 : X_1 : X_2 : X_3] \in D$, then we see similarly to part a) that:
\[         X_3^2 = - X_0^2 - X_1^2 - X_2^2     \]
\[          X_3^3 = - X_0^3 - X_1^3 - X_2^3   \]
\[         F(X_0, X_1, X_2) =    ( X_0^2 + X_1^2)^3 + (X_0^3 + X_1^3)^2 = 0        \]
We take $D'$ to be defined by $F = 0$. Now, given $X_0, X_1, X_2$, we attempt to construct the preimage $[X_0 : X_1 : X_2 : X_3] \in D$ by:
\[       X_3 = \frac{X_0^3 + X_1^3 + X_2^3}{X_0^2 + X_1^2 + X_2^2}         \]
If this is well-defined, i.e. for $X_0^2 + X_1^2 + X_2^2 \neq 0$, it is the unique point in the preimage. If this is not defined, we have $X_0^2 + X_1^2 + X_2^2 = X_0^3 + X_1^3 + X_2^3 = 0$, and we are in the situation of part a), i.e. there are 6 points in the preimage.
\\
\\
c) The degree of $D'$ is the degree of $F$, which is 6.
\\
\\
d) \[       \frac{\p F}{\p X_i} = 2 (\sum X_j^3) 3 X_i^2 + 3 \left(\sum X_j^2\right)^2 2X_i = 0     \]
Summing over all $i$ we get:
\[         6 \left(\sum X_i^2\right) \left[ \sum X_i^3 + \left( \sum X_i^2\right) \left( \sum X_i \right) \right]  = 0    \]
Using Mathematica, we see that the different ways in which this can be solved in $\Proj^2$ break down to the case $\sum X_i^2 = \sum X_i^3 = 0$, and by part a) there are 6 solutions to this. Therefore $D'$ has 6 singular points.
\\
\\
e) We guess that each of the 6 singularities decreases the genus by exactly 1, and then the genus of $D'$ is $g = 10 - 6 = 4$. Then, since $D$ and $D'$ are birational, we expect that the genus of $D$ is also 4.




\end{document}


































