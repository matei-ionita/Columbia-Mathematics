\documentclass[12 pt]{article}
\usepackage{amsmath,amssymb,amsthm,fullpage,amsfonts,enumerate,textcomp, eurosym, tikz-cd, fullpage}
\title{Commutative algebra HW2}
\author{Matei Ionita}

\DeclareMathOperator {\p} {\partial}
\DeclareMathOperator {\R} {\mathbb{R}}
\DeclareMathOperator {\C} {\mathbb{C}}
\DeclareMathOperator {\Q} {\mathbb{Q}}
\DeclareMathOperator {\Z} {\mathbb{Z}}
\DeclareMathOperator {\fr} {\mathfrak}


\begin{document}
  \maketitle

\subsection*{Problem 1}
Let $k$ be a field. Let $k[[t]]$ be the power series ring over $k$. Show that $k[[t]]$ is a local ring.
\\
\\
\emph{Solution}
\\
We can show that any power series with nonzero constant coefficient is invertible; thus the ideal $(t)$ contains all non-units, and $\big(k[[t]], (t)\big)$ is a local ring. Consider:
\[    F(t) = \left( a_0 + \sum_{n=1}^{\infty} a_n t^n  \right)^{-1}       \]
Then taking formal derivatives we can write $F(t) = \sum_{n=0}^{\infty} F^{(n)}(0) t^n / n!$, so $F(t) \in k[[t]]$, and we found an inverse for the original power series. This works as long as $F^{(n)} (0)$ doesn't have a zero denominator, which is always the case as long as $a_0 \neq 0$.

\subsection*{Problem 2}
Give an example of (1) a local ring with 2 prime ideals and (2) a local ring with 3 prime ideals.
\\
\\
\emph{Solution}
\\
(1) Take $k[[t]]$ as in problem 1, its only primes are $(0)$ and $(t)$.
\\
(2) Take $\C[x,y]/(xy)$. In HW1 we saw that its primes are $(x), (y), (x-\lambda, y), (x, y-\mu)$. If we localize this ring at $(x,y)$, the only remaining primes will be the ones contained in $(x,y)$, which are $(x), (y), (x,y)$.

\subsection*{Problem 3}
Let $R = k[[t]]$ where $k$ is a field. Give an example of a module $M$ over $R$ such that $M = tM$ (in other words, a module which contradicts the conclusion of Nakayama's lemma).
\\
\\
\emph{Solution}
\\
Consider $M = k[[t, t^{-1}]]$, which is an infinitely generated module over $k[[t]]$, with generators $t^{-n}$. Any element of $M$ looks like $a = \sum_{i=-\infty}^{\infty} k_i t^i$. Then $ta = \sum_{i=-\infty}^{\infty} k_{i-1} t^i$, so $tM\subset M$. Moreover, for all $a$ there exists some $b\in M$ such that $a = bt$, namely $b = \sum_{i=-\infty}^{\infty} k_{i+1} t^i$. Therefore $M\subset tM$. So $M = tM$.

\subsection*{Problem 4}
Let $R = \C[x]$ be the polynomial ring over the complex numbers. Let $m_n$, $n = 1, 2, 3, ...$ be an infinite sequence of pairwise distinct maximal ideals of $R$. Show that $R$ does not surject onto the product of the rings $R/m_n$ (contradicting the conclusion of the Chinese remainder theorem).
\\
\\
\emph{Solution}
\\
All maximal ideals of $R$ are of the form $m_i = (z-\lambda_i)$, therefore $R/m_i = \C[x]/(z-\lambda_i) \cong \C$. Assume then that there exists a surjection
\[     \phi: \C[x] \to \C \times \C \times \dots   \]
Any homomorphism preserves $\C \subset \C[x]$, so we impose $\phi(\lambda) = (\lambda, \lambda, \dots )$ for all $\lambda\in \C$. Because $\phi$ is surjective, there exists $a\in \C[x]$ such that $\phi(a) = (1, 0 , 0, \dots)$. But this means that $a\in m_i$ for all $i\geq 2$, so $a = f(x) (x-\lambda_2)(x-\lambda_3) \dots$ for some $f\in \C[x]$. $a$ has an infinite number of roots, so $a=0$. But this contradicts the fact that $\phi(0) = (0, 0, \dots)$. Therefore there exists no surjection $\C[x] \to \C \times \C \times \dots$

\subsection*{Problem 5}
Let $k$ be a field. Find the minimal prime ideals of $k[x, y, z]/(xy, xz, yz)$.
\\
\\
\emph{Solution}
\\
We are looking for primes of $k[x,y,z]$ that contain $(xy), (xz), (yz)$. The possible generators for primes in $k[x,y,z]$ are $x-\lambda, y-\lambda, z-\lambda$ as well as irreducible higher order polynomials $f$ in $x,y,z$. But we do not care about the higher order polynomials, since these do not generate $xy, xz$ or $yz$, and therefore any prime that contains $f$ will contain another prime ideal which generates $xy, xz, yz$ without the help of $f$. Therefore the primes which contain $f$ will not be minimal. Similarly, we do not care about $x-\lambda, y-\lambda, z-\lambda$ for nonzero $\lambda$, since these do not help generate $xy, xz, yz$ either, and they will prevent ideals from being minimal. All primes of interest are therefore generated by $x, y, z$; the possible ones are:
\[      (x,y) \;\;\; (x,z) \;\;\; (y,z) \;\;\; (x,y,z)      \]
But $(x,y,z)$ contains all other three primes, and thus the only minimal primes are $(x,y), (y,z), (x,z)$.


\end{document}







































