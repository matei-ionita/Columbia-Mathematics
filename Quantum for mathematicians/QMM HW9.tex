\documentclass[12 pt]{article}
\usepackage{amsmath,amssymb,amsthm,fullpage,amsfonts,enumerate,textcomp, eurosym}
\usepackage{yfonts}
\usepackage[T1]{fontenc}
\title{QM for Mathematicians HW9}
\author{Matei Ionita}


\DeclareMathOperator {\p} {\partial}
\DeclareMathOperator {\R} {\mathbb{R}}
\DeclareMathOperator {\C} {\mathbb{C}}
\DeclareMathOperator {\Q} {\mathbb{Q}}
\DeclareMathOperator {\Z} {\mathbb{Z}}
\DeclareMathOperator {\lan} {\langle}
\DeclareMathOperator {\ran} {\rangle}

\begin{document}
  \maketitle


\subsection*{Problem 1}
Let $\Lambda = e^{tK_1}$; the commutator of Lie algebra elements is:
\[     [K_1 , P_m ] = \frac{d}{dt} (\Lambda P_m \Lambda^{-1}) |_{t=0}      \]
For some translation $(a,1)$, using the group law gives:
\[        (0,\Lambda) (a,1) (0,\Lambda^{-1}) =  (\Lambda a , \Lambda) (0, \Lambda^{-1}) = (\Lambda a, 1)    \]
Therefore:
\[        [K_1, P] =    \frac{d}{dt} (\Lambda P) |_{t=0} = \frac{d}{dt}  \left( \begin{array} {cccc} 
\cosh t & \sinh t & & \\ 
\sinh t & \cosh t & & \\
 & & 1 & \\
 & & & 1  
\end{array} \right) \left(  \begin{array} {c} P_0 \\ P_1 \\ P_2 \\ P_3   \end{array}   \right)     \]
\[         [K_1, P] =  \left( \begin{array} {cccc} 
 & 1 & & \\ 
1 &  & & \\
 & & & \\
 & & & 
\end{array} \right) \left(  \begin{array} {c} P_0 \\ P_1 \\ P_2 \\ P_3   \end{array}   \right)   =   \left(  \begin{array} {c} P_1 \\ P_0 \\ 0 \\ 0   \end{array}   \right)     \]
Therefore:
\[   [K_1, P_0] = P_1   \;\;\;  [K_1, P_1] = P_0 \;\;\; [K_1, P_2] = 0 \;\;\; [K_1, P_3] = 0    \]
In general:
\[        [K_l , P_0] = P_l   \;\;\;\;\; [K_l, P_m] = \delta_{lm} P_0 \text{ , for } m\neq 0   \]

\subsection*{Problem 2}
a) \[      \hat P = -i \int d^3 x \hat \Pi (x) ( - i \nabla) \hat \phi(x)        \]
\[         \hat \Pi (x) = \dot \phi (x) = \int \frac{d^3 k}{(2\pi)^{3/2} (2\omega_k)^{1/2}} i \omega_k \left( -  a_k e^{-ikx} + a_k^{\dagger} e^{ikx}  \right)       \]
\[      \hat P = - \int \frac{d^3 k d^3 k'}{(2\pi)^3}  \frac{1}{2}   \int d^3 x \left(  a_k e^{-ikx} - a_k^{\dagger} e^{ikx}  \right) \mathbf{k'} \left(  a_{k'} e^{-ik'x} - a_{k'}^{\dagger} e^{ik'x}  \right)   \]
\[        \hat P =   \int \frac{d^3 k d^3 k'}{2 (2\pi)^3} \mathbf{k'} \int d^3 x \left[ (a_k a_{k'}^{\dagger} + a_k^{\dagger} a_{k'} ) e^{i(k-k')x} - (a_k a_{k'} + a_k^{\dagger} a_{k'}^{\dagger} ) e^{i(k+k')x} \right]   \]
\[        \hat P = \int  \frac{d^3 k d^3 k'}{2 (2\pi)^3} \mathbf{k'} \left[  (a_k a_{k'}^{\dagger} + a_k^{\dagger} a_{k'} ) \delta(k-k') - (a_k a_{k'} + a_k^{\dagger} a_{k'}^{\dagger} ) \delta(k+k')    \right]        \]
\[          \hat P = \int   \frac{d^3 k}{2 (2\pi)^3} \mathbf{k}   (a_k a_{k}^{\dagger} + a_k^{\dagger} a_{k} )  + \int  \frac{d^3 k}{2 (2\pi)^3} \mathbf{k}  (a_k a_{-k} + a_k^{\dagger} a_{-k}^{\dagger} )      \]
Note that, since $[a_k, a_{-k}] = 0$ and $[a_k^{\dagger} , a_{-k}^{\dagger} ] = 0$, the integrand in the second term is even, and thus the integral is 0. In the first term, we use the highly suspicious procedure of normal ordering to obtain:
\[        \hat P =     \int   \frac{d^3 k}{ (2\pi)^3} \mathbf{k} \;  a_k a_{k}^{\dagger}      \]
\\
\\
b) \[        \hat P = -i \int d^3 x \hat \Pi(x) (-i \nabla) \hat \phi(x) + \text{h. c.}  = -i \int d^3 x \left[   \dot \phi^{\dagger}(x) (-i) \nabla \phi(x) + \nabla \phi^{\dagger} (x) (i) \dot \phi(x)  \right]   \]
We use:
\[         \phi(x) = \int \frac{d^3 k}{(2\pi)^{3/2} (2\omega_k)^{1/2}} \left(  a_k e^{-ikx} + b_k^{\dagger} e^{ikx}  \right)   \]
\[         \phi^{\dagger} (x) = \int \frac{d^3 k}{(2\pi)^{3/2} (2\omega_{k'})^{1/2}} \left(  b_{k'} e^{-ik'x} + a_{k'}^{\dagger} e^{ik'x}  \right)                        \]
Let's focus on the first term:
\[      -  \int \frac{d^3 k d^3 k'}{(2\pi)^3 }  \mathbf{k'} \int d^3 x (b_{k'} e^{-ik'x} - a_{k'}^{\dagger} e^{ik'x}) (  a_k e^{-ikx} - b_k^{\dagger} e^{ikx} )   \]
\[  -  \int \frac{d^3 k d^3 k'}{(2\pi)^3 }  \mathbf{k'} \int d^3 x \left[   (b_{k'}a_k + a_{k'}^{\dagger}b_k^{\dagger}) e^{i(k+k')x} - (b_{k'} b_k^{\dagger} + a_{k'}^{\dagger} a_k) e^{i(k-k') x}  \right]              \]
\[      \int d^3 k \frac{1}{2}    \mathbf{k}  \left[ -  (b_{-k}a_k + a_{-k}^{\dagger}b_k^{\dagger}) + (b_{k} b_k^{\dagger} + a_{k}^{\dagger} a_k)  \right]       \]
Doing the same computation, we get the second term:
\[       \int d^3 k \frac{1}{2}    \mathbf{k}  \left[ -  (a_{-k}b_k + b_{-k}^{\dagger}a_k^{\dagger}) + (b_{k}^{\dagger} b_k + a_{k} a_k^{\dagger})  \right]             
 \]
Adding them together gives:
\[      \hat P = \int d^3 k \frac{1}{2} \mathbf{k} \left[   -  ( a_{-k} b_{k} + b_{-k} a_{k} +  a^{\dagger}_{-k} b^{\dagger}_{k} + b^{\dagger}_{-k} a^{\dagger}_{k}  ) + ( b_{k} b_k^{\dagger} +  b^{\dagger}_{k} b_k + a_{k}^{\dagger} a_k + a_{k} a^{\dagger}_k )  \right]      \]
By the same argument as before, the first term is odd, so its integral is 0. In the second term we use normal ordering to get:
\[         \hat P = \int d^3 k \mathbf{k} (  b_{k} b_k^{\dagger} + a_{k} a^{\dagger}_k  )       \]

\subsection*{Problem 3}
The Lagrangian for a free theory with two complex fields is:
\[      \mathcal{L} = - \p^{\mu} \phi^{\dagger} \p_{\mu} \phi -  \p^{\mu} \psi^{\dagger} \p_{\mu} \psi - m^2 \phi^{\dagger} \phi - m^2 \psi^{\dagger} \psi  \]
We need some linear transformation:
\[        \left( \begin{array} {c} \phi ' \\ \psi' \end{array} \right) =  U  \left( \begin{array} {c} \phi  \\ \psi \end{array} \right)    \]
Such that the Lagrangian is invariant, i.e. $\phi^{\dagger} \phi + \psi^{\dagger} \psi$ is unchanged. This can be written as:
\[          \left( \begin{array} {cc} \phi ' & \psi' \end{array} \right)  \left( \begin{array} {c} \phi ' \\ \psi' \end{array} \right)   =   \left( \begin{array} {cc} \phi  & \psi \end{array} \right) U^{\dagger} U \left( \begin{array} {c} \phi  \\ \psi \end{array} \right) = \left( \begin{array} {cc} \phi  & \psi \end{array} \right)  \left( \begin{array} {c} \phi  \\ \psi \end{array} \right)     \]
Therefore we need  to impose $U^{\dagger} U = I_2$, so $U\in U(2)$. To find a basis for $\mathfrak{u}(2)$, note that this Lie algebra consists of skew-Hermitian matrices, so we impose:
\[     \left( \begin{array} {cc}  a e^{i\alpha} & b e^{i \beta} \\ c e^{i\gamma} & d e^{i\delta}  \end{array}   \right)   = -  \left( \begin{array} {cc}  a e^{-i\alpha} & c e^{-i\gamma}  \\ b e^{-i \beta} & d e^{-i\delta}  \end{array}   \right)   \]
Where a, b, c, d are real. Thus, the most general form of a $\mathfrak{u}(2)$ matrix is:
\[          \left( \begin{array} {cc}  a i & -c e^{-i \gamma} \\  c e^{i\gamma} & d i  \end{array}   \right)  =  a \left( \begin{array} {cc}  i & 0 \\ 0 & 0  \end{array}   \right)   +   d \left( \begin{array} {cc}  0 & 0 \\ 0 & i  \end{array}   \right)   + c  \left( \begin{array} {cc}  0 & - e^{-i \gamma} \\  e^{i\gamma} & 0  \end{array}   \right)    =     \]
\[  =       a \left( \begin{array} {cc}  i & 0 \\ 0 & 0  \end{array}   \right)   +   d \left( \begin{array} {cc}  0 & 0 \\ 0 & i  \end{array}   \right)   + c \cos \gamma  \left( \begin{array} {cc}  0 & - 1 \\  1 & 0  \end{array}   \right)   + c \sin \gamma \left( \begin{array} {cc}  0 & i \\  i & 0  \end{array}   \right)  \]
We have found a basis for $\mathfrak{u}(2)$. Let's check what symmetries the four basis elements generate:
\[     \text{Exp }  \left( \begin{array} {cc}  ai & 0 \\ 0 & 0  \end{array}   \right) =   \left( \begin{array} {cc}  e^{ia} & 0 \\ 0 & 0  \end{array}   \right)   \]
This performs a $U(1)$ tranformation on $\phi$. Similarly:
\[     \text{Exp }  \left( \begin{array} {cc}  0 & 0 \\ 0 & di  \end{array}   \right) =   \left( \begin{array} {cc}  0 & 0 \\ 0 & e^{id}  \end{array}   \right)   \]
Which performs a $U(1)$ tranformation on $\psi$. Then:
\[             \text{Exp }  \left( \begin{array} {cc}  0 & -b \\ b & 0  \end{array}   \right) =   \left( \begin{array} {cc}  \cos b & -\sin b \\ \sin b & \cos b  \end{array}   \right)                      \]
Which performs an $SO(2)$ rotation of the two fields. Finally:
\[           \text{Exp }  \left( \begin{array} {cc}  0 & c i \\ c i & 0  \end{array}   \right) =   \left( \begin{array} {cc}  \cos c & i \sin c \\ i \sin c & \cos c  \end{array}   \right)          \]
Therefore we can use this basis as our charge operators:
\[           \left( \begin{array} {cc}  i & 0 \\ 0 & 0  \end{array}   \right)  \;\;\;  \left( \begin{array} {cc}  0 & 0 \\ 0 & i  \end{array}   \right)  \;\;\;  \left( \begin{array} {cc}  0 & - 1 \\  1 & 0  \end{array}   \right)   \;\;\; \left( \begin{array} {cc}  0 & i \\  i & 0  \end{array}   \right)       \]























\end{document}









